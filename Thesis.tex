%%%%%%%%%%%%%%%%%%%%%%%%%%%%%%%%%%%%%%%%%
% Masters/Doctoral Thesis 
% LaTeX Template
% Version 1.43 (17/5/14)
%
% This template has been downloaded from:
% http://www.LaTeXTemplates.com
%
% Original authors:
% Steven Gunn 
% http://users.ecs.soton.ac.uk/srg/softwaretools/document/templates/
% and
% Sunil Patel
% http://www.sunilpatel.co.uk/thesis-template/
%
% License:
% CC BY-NC-SA 3.0 (http://creativecommons.org/licenses/by-nc-sa/3.0/)
%
% Note:
% Make sure to edit document variables in the Thesis.cls file
%
%%%%%%%%%%%%%%%%%%%%%%%%%%%%%%%%%%%%%%%%%

%----------------------------------------------------------------------------------------
%	PACKAGES AND OTHER DOCUMENT CONFIGURATIONS
%----------------------------------------------------------------------------------------

\documentclass[11pt, a4paper]{Thesis} % Paper size, default font size and one-sided paper

\graphicspath{{Pictures/}} % Specifies the directory where pictures are stored

\usepackage[square, numbers, comma, sort&compress]{natbib} % Use the natbib reference package - read up on this to edit the reference style; if you want text (e.g. Smith et al., 2012) for the in-text references (instead of numbers), remove 'numbers' 
\usepackage{makeidx}
\makeindex
\usepackage{appendix}
\usepackage[all]{xy}
\hypersetup{urlcolor=blue, colorlinks=true} % Colors hyperlinks in blue - change to black if annoying
\title{\ttitle} % Defines the thesis title - don't touch this

\begin{document}

\frontmatter % Use roman page numbering style (i, ii, iii, iv...) for the pre-content pages

\setstretch{1.3} % Line spacing of 1.3

% Define the page headers using the FancyHdr package and set up for one-sided printing
\fancyhead{} % Clears all page headers and footers
\rhead{\thepage} % Sets the right side header to show the page number
\lhead{} % Clears the left side page header

\pagestyle{fancy} % Finally, use the "fancy" page style to implement the FancyHdr headers

\newcommand{\HRule}{\rule{\linewidth}{0.5mm}} % New command to make the lines in the title page

% PDF meta-data
\hypersetup{pdftitle={\ttitle}}
\hypersetup{pdfsubject=\subjectname}
\hypersetup{pdfauthor=\authornames}
\hypersetup{pdfkeywords=\keywordnames}

%----------------------------------------------------------------------------------------
%	TITLE PAGE
%----------------------------------------------------------------------------------------

\begin{titlepage}
\begin{center}

{\huge \bfseries \ttitle}\\[0.4cm] % Thesis title

\textsc{\Large PROEFSCHRIFT}\\[0.5cm] % Thesis type

\large \textit{ter verkrijgen van\\
de graad van \degreename{} aan de \univname,\\
op gezag van de Rector Magnificus Prof. mr. dr. Carel Stolker,\\
volgens besluit van het College voor Promoties\\
te verdedigen op ...\\
klokke ... uur}

\large \textit{door}

\authornames

\large \textit{geboren te Oud-Beijerland op 9 mei 1978}
 
\vfill
\end{center}

\end{titlepage}

%----------------------------------------------------------------------------------------
%	PROMOTIECOMMISSIE
%----------------------------------------------------------------------------------------

\begin{titlepage}
PROMOTIECOMMISSIE

\textit{Promotor}\\
Prof. dr. J.N. Kok

\textit{Copromotor}\\
Dr. Ir. F.J. Verbeek

\textit{Overige Leden}\\
?\\
?\\
?\\
?
\vfill
This research was supported by the Cyttron project
\end{titlepage}

\clearpage % Start a new page

%----------------------------------------------------------------------------------------
%	DEDICATION
%----------------------------------------------------------------------------------------

\setstretch{1.3} % Return the line spacing back to 1.3

\pagestyle{empty} % Page style needs to be empty for this page

\dedicatory{This thesis is dedicated in loving memory of my
  parents\\ Bas and Marja. } % Dedication text

\addtocontents{toc}{\vspace{2em}} % Add a gap in the Contents, for aesthetics


%----------------------------------------------------------------------------------------
%	QUOTATION PAGE
%----------------------------------------------------------------------------------------

\pagestyle{empty} % No headers or footers for the following pages

\null\vfill % Add some space to move the quote down the page a bit

\textit{``Als je een geleerd iemand bent, moet je lang nadenken over een vraag en moet je een zwaar boek erbij pakken; dat denkt het beste."}

\begin{flushright}
Eponine de Beer (7 jaar)
\end{flushright}

\vfill\vfill\vfill\vfill\vfill\vfill\null % Add some space at the bottom to position the quote just right

\clearpage % Start a new page

%----------------------------------------------------------------------------------------
%	LIST OF CONTENTS/FIGURES/TABLES PAGES
%----------------------------------------------------------------------------------------

\pagestyle{fancy} % The page style headers have been "empty" all this time, now use the "fancy" headers as defined before to bring them back

\lhead{\emph{Contents}} % Set the left side page header to "Contents"
\tableofcontents % Write out the Table of Contents

\lhead{\emph{List of Figures}} % Set the left side page header to "List of Figures"
\listoffigures % Write out the List of Figures

\lhead{\emph{List of Tables}} % Set the left side page header to "List of Tables"
\listoftables % Write out the List of Tables

%----------------------------------------------------------------------------------------
%	ABBREVIATIONS
%----------------------------------------------------------------------------------------

\clearpage % Start a new page

\setstretch{1.5} % Set the line spacing to 1.5, this makes the following tables easier to read

\lhead{\emph{Abbreviations}} % Set the left side page header to "Abbreviations"
\listofsymbols{ll} % Include a list of Abbreviations (a table of two columns)
{
\textbf{DL} & \textbf{D}escription \textbf{L}ogic \\
\textbf{FOL} & \textbf{F}irst-\textbf{O}rder \textbf{L}ogic \\
\textbf{OWL} & \textbf{W}eb \textbf{O}ntology \textbf{L}anguage \\
\textbf{RDF} & \textbf{R}esource \textbf{D}escription \textbf{F}ramework \\
}

%----------------------------------------------------------------------------------------
%	SYMBOLS
%----------------------------------------------------------------------------------------

\clearpage % Start a new page

\lhead{\emph{Symbols}} % Set the left side page header to "Symbols"

\listofnomenclature{ll} % Include a list of Symbols (a two column table)
{
\(\top\) & Top \\ 
\(\bot\) & Bottom \\ 
\(\sqsubseteq\) & Concept Inclusion \\
\(\sqcup\) & Union of Concepts \\
\(\sqcap\) & Intersection of Concepts \\
\(\neg\) & Negation of Concepts \\
\(\forall\) & Universal Restriction \\
\(\exists\) & Existential Restriction \\
\(\equiv\) & Concept Equivalence \\
\(\doteq\) & Concept Definition \\
\(\colon\) & Concept or Role Assertion \\
% Symbol & Name \\
}

%----------------------------------------------------------------------------------------
%	THESIS CONTENT - CHAPTERS
%----------------------------------------------------------------------------------------

\mainmatter % Begin numeric (1,2,3...) page numbering

\pagestyle{fancy} % Return the page headers back to the "fancy" style

% Include the chapters of the thesis as separate files from the Chapters folder
% Uncomment the lines as you write the chapters

% Chapter Template

\chapter{Introduction} % Main chapter title

\label{Chapter1}

\lhead{Chapter 1. \emph{Introduction}} 

%------------------------------------------------------------------------------
%	Overview
%------------------------------------------------------------------------------

\section{General Overview}

The scientific method \index{Scientific method} relies on the
generation of measurements\index{Measurements}. For reproducability
\index{Reproducability} measurements should be reported in such a way
that other scientist can replicate the results.

Within computational science the ability to replicate results is
called Reproducable Research\index{Reproducable
  research}\cite{FomelReproducibleResearch}. For the Life
Sciences\index{Life science} this approach is more problematic,
because of the enormous amount of parameters in any given
experiment. A simple workflow\index{Workflow} might look like this:

\begin{itemize}
\item Preperation of sample\index{Sample preperation}
\item Microscopy on sample\index{Microscopy}
\item Image analysis\index{Image analysis}
\item Statistical analysis\index{Statistical analysis}
\end{itemize}

Each of these steps deserve a methods page on their own, yet papers
need to communicate their findings succinctly. When there is no room
for in the papers for the provenance\index{Provenance} information,
this information has to stored next to or inside the data itself.

In this thesis I will explore the questions how we can capture, store
and leverage this metadata\index{Metadata}, especially in the area of
microscopy images\index{Microscopy images}.


\subsection{Contextual information}

When talking about microscopy images\index{Microscopy images}, it is
not enough to just share the pixels\index{Pixels} of the
image. Without contextual information\index{Contextual information}
the data is useless.

How do we measure the quality of the contextual information of a
microscope image? To get an answer, we take three steps:
\begin{itemize}
\item Analytical decomposition\index{Analytical decomposition} of the
  pipeline\index{pipeline} of data acquisition\index{Data acquisition}
  in microscopy science\index{Microscopy science}.
\item Survey experimental biologists\index{Experimental biologist},
  bio-informaticians\index{Bio-informatician}, theoretical
  biologists\index{Theoretical biologist} and medical
  experts\index{Medical expert}.
\item Check existing image repositories\index{Image repository} for
  their metadata\index{Metadata}.
\end{itemize}

I assume the following elements need to be described:
\begin{itemize}
\item Biological origin\index{Biological origin}:
  Species\index{Species}/Cell culture\index{Cell culture} (including
  age), Anatomy structures\index{Anatomy structures}
\item Chemical and physical conditions\index{Chemical
  conditions}\index{Physical conditions}
\item Imaging techniques\index{Imaging techniques}
\end{itemize}

%------------------------------------------------------------------------------
%	Semantics and Reasoning
%------------------------------------------------------------------------------

\section{Semantics and Reasoning}

This section describes the main technology that is used to arive at
new insights. What do I mean when I talk about semantics and
reasoning? This describes the larger field, with a subsection delving
into the semantic web.

%-----------------------------------
%	Semantic Web
%-----------------------------------

\subsection{Semantic Web}

This section describes the Semantic Web\index{Semantic web} idea and
its main technologies. What are the restrictions, what can we expect?

The Semantic Web is a concept that revolves around the usage of
different technologies. One of the most important languages used is
that of OWL (Web Ontology Language)\index{OWL}. In this thesis we will
be using the latest version of OWL at the time of writing, which is
OWL 2. we will use Manchester Syntax\index{Manchester Syntax} to
describe the formal structure of our
ontologies\index{Ontology}. Because we want full flexibility in our
modeling\index{Modeling}, we will use OWL 2 Full by using RDF-based
semantics\index{RDF}\index{semantics}.



%------------------------------------------------------------------------------
%	Annotation and Retrieval
%------------------------------------------------------------------------------

\section{Annotation and Retrieval}

This section will provide information about
annotation\index{Annotation} of other resources, images in
particular. The classical use for annotation is
retrieval\index{Retrieval}. Annotation may also be used for
reasoning\index{Reasoning} and pattern discovery\index{Pattern
  discovery}.

%------------------------------------------------------------------------------
%	Life Science Images
%------------------------------------------------------------------------------

\section{Life Science Images}


%-----------------------------------
%	Developmental biology and model organisms
%-----------------------------------

\subsection{Developmental biology and model organisms}


%-----------------------------------
%	Image Acquisition in the Life Sciences
%-----------------------------------

\subsection{Image Acquisition in the Life Sciences}



% Chapter Template

\chapter{Image Knowledge Systems} % Main chapter title

\label{Chapter2} % Change X to a consecutive number; for referencing this chapter elsewhere, use \ref{ChapterX}

\lhead{Chapter 2. \emph{Image Knowledge Systems}} % Change X to a consecutive number; this is for the header on each page - perhaps a shortened title

%------------------------------------------------------------------------------
%	Image metadata
%------------------------------------------------------------------------------

\section{Image Metadata}

Image metadata\index{Metadata!Image|textbf} includes structural
metadata\index{Metadata!Structural}, descriptive
metadata\index{Metadata!Descriptive} and administrative
metadata\index{Metadata!Administrative}. An important part of the
metadata is provenance\index{Provenance} information.

%------------------------------------------------------------------------------
%	Image Repositories
%------------------------------------------------------------------------------

\section{Image Repositories}

Image Repositories\index{Image!Repository|textbf} are places where
images can be stored and retrieved. They can differ in functionality
and range from simple FTP-servers\index{FTP-server} to full
semantic\index{Semantics} image repositories.
 
% Chapter Template

\chapter{Ontologies for Bioimages} % Main chapter title

\label{Chapter 3} % Change X to a consecutive number; for referencing this chapter elsewhere, use \ref{ChapterX}

\lhead{Chapter 3. \emph{Ontologies for Bioimages}} % Change X to a consecutive number; this is for the header on each page - perhaps a shortened title

%------------------------------------------------------------------------------
%	Imaging Ontology
%------------------------------------------------------------------------------

\section{Imaging Ontology}

The microscopy lab at Leiden University both contains commercial
standardized microscopes and specialized custom built
microscopes. Experimental biologists need the flexibility to customize
their equipment to find new ways to test their hypotheses. The images
that were acquired are often sent to image analysts to gain statistics
from their measurements. The image analysis often requires a good
understanding of the exact methods by which the images were
acquired. Moreover, when sharing the data after publication, it is
important to have correct provenance information.

A tried solution has been the creation of imaging ontologies. These
ontologies were mostly created by single labs which focus on their
particular use-cases. Although most of the labs take care to abstract
and broaden the scope of their particular ontology, this model will
not scale. In practice any additions that an outside lab wants to make
to the ontology will be reviewed with the goals of the original lab in
mind.

We propose an client-server architecture for our imaging ontology. In
this architecture, there will be a central ontology, called the master
ontology, that holds the biggest subset of statements that all the
labs can agree on. Every lab will have their own local imaging
ontology that extends the master ontology.

Local ontologies will be placed inside their own local namespace. This
ensures that no name clashes will occur between different local
ontologies. Images are always annotated according to a local
ontology. When an image is shared, the local classes are transformed
into their most specific master ontology parent class. This allows for
the most semantically relevant information to be shared, while keeping
any secret microscope information hidden.

The extension of the local ontologies will be driven by a user
interface that limits the addition of rules that are consistent with
the current master and local ontology. Statements from all the local
ontologies are aggregated and inspected by domain and ontology experts
who will update the master ontology.

The update process proceeds along the following steps:
\begin{itemize}
  \item A pattern is observed from local ontologies
  \item A domain expert describes the changes that have to be made
  \item An ontology expert constructs the rules to transform the ontologies
  \item The transformation is tested on the master and local ontologies and validated by reasoners
  \item Any validation errors are resolved by the ontology expert or the change is postponed until consensus is reached
  \item The validated changes are put into place
\end{itemize}

The technique used here is best described by the software pattern
called the observer pattern. Using the terminology of the observer
pattern, the master ontology is the subject and the local ontologies
are observers. The local ontologies are notified when the master
ontology requires local changes.

The transformations on the master and local ontologies are sent using
SPARQL\index{SPARQL} 1.1 constructs. In this version of SPARQL,
inserts and deletions are possible. In our initial setup the local
ontologies are located on the same system as the master ontology, but
nothing in this architecture requires the local ontology to be on the
same system.

To start this updating process we have to construct a reasonable first
master version of the imaging ontology. The imaging
ontology\index{Imaging~ontology|see{Ontology,~Imaging}}\index{Ontology!Imaging|textbf}
describes imaging equipment\index{Imaging equipment} and their related
modalities\index{Modality}. It builds on PROV\index{PROV} and includes
semantics\index{Semantics} to facilitate reasoning\index{Reasoning}
over microscope types.



%------------------------------------------------------------------------------
%	Staging Ontology
%------------------------------------------------------------------------------

\section{Staging Ontology}

In animal experiments\index{Animal experiments} it is important to
know the age of the subject. Many experiments are done in
developmental stages\index{Developmental stages}. Absolute time
measurement doesn't make sense when comparing different species or
sometimes even specimens within a single species. Uncontrollable
external factors can determine the speed at which an organism
grows. Various staging descriptions have been developed for the varied
model organisms that are used in experiments. Among them are the
following staging systems:

\begin{itemize}
\item Carnegie Stages\cite{CarnegieStage}\index{Stages!Carnegie}
  (Vertebrate Embryos)
\item Hamburger-Hamilton
  Stages\cite{HamburgerHamiltonStage}\index{Stages!Hamburger-Hamilton}
  (Chicken Embryos)
\item Hisoaka Battle
  Stages\cite{HisaokaBattleStage}\index{Stages!Hisoaka Battle}
  (Zebrafish Embryos)
\item Kimmel Stages\cite{KimmelStage}\index{Stages!Kimmel} (Zebrafish
  Embryos)
\item Theiler Stages\cite{TheilerStage}\index{Stages!Theiler} (Mouse
  Embryos)
\item Yntema Stages\cite{YntemaStage}\index{Stages!Yntema} (Turtle
  Embryos)
\end{itemize}

All these stages are defined in the Staging
Ontology\index{Staging~Ontology|see{Ontology,~Staging}}\index{Ontology!Staging|textbf}. Although
time has been modelled in the past in OWL, the main domain was regular
date and clock time. In the biological domain, we see other patterns
to describe the temporal state of living organisms. In embryology
staging schemes are used to describe the development of a
subject. These staging schemes usually rely on visual characteristics
of the subject to determine the progression along the maturation
process. To capture this sense of time, we developed the staging
ontology, which combines several staging schemes together, expressed
in OWL and quantifies them using properties and creates parallels
using relations.


%------------------------------------------------------------------------------
%	Biological Relative Position Ontology
%------------------------------------------------------------------------------

\section{Biological Relative Position Ontology}

The Biological Relative Position
Ontology\index{Biological~Relative~Position~Ontology|see{Ontology,~Biological~Relative~Position}}\index{Ontology!Biological~Relative~Position|textbf}
is an ontology\index{Ontology} that describes positions between
entities. These spatial relations have been described in the [TODO:
  look up reference]. We used this ontology and show that it can be
used to characterize and reason over spatial data. In our use case we
made use of TDR-3D to create 3D models from microscopy images. These
models are then parsed and automatically converted into a logical
spatial model. This model can then be used for validation and
querying.

% Chapter Template

\chapter{Use Case: Cyttron Database} % Main chapter title

\label{Chapter4} % Change X to a consecutive number; for referencing this chapter elsewhere, use \ref{ChapterX}

\lhead{Chapter 4. \emph{Use Case: Cyttron Database}} % Change X to a consecutive number; this is for the header on each page - perhaps a shortened title

%------------------------------------------------------------------------------
%	Cyttron overview
%------------------------------------------------------------------------------

\section{Cyttron overview}
The Cyttron Database\index{Cyttron database} is an image database that uses
semantic keywords to annotate images.

%------------------------------------------------------------------------------
%	Annotation for Microscopes
%------------------------------------------------------------------------------

\section{Annotation for Microscopes}

In my research I have explored how we can augment the data we have on
the creation on the images in Cyttron\index{Cyttron database}. During
the
acquisition\index{Aquisition|see{Image,~Acquisition}}\index{Image!Acquisition}
of images, metadata\index{Metadata} needs to be gathered and
translated into a common integrated schema\index{Schema}. The
microscope\index{Microscope} that was used for the acquisition plays
an important role when trying to understand the meaning of the image
data.

%-----------------------------------
%	Annotating old data
%-----------------------------------

\subsection{Annotating old data}

In this section we explore how old data can be retroactively be
annotated\index{Annotating!old data}.

%-----------------------------------
%	Annotating new data
%-----------------------------------

\subsection{Annotating new data}

In this section we explore how new data can be most effectively be
annotated\index{Annotating!new data}.
 
% Chapter Template

\chapter{Use Case: Cytomics Database} % Main chapter title

\label{Chapter5} % Change X to a consecutive number; for referencing this chapter elsewhere, use \ref{ChapterX}

\lhead{Chapter 5. \emph{Cytomics Database}} % Change X to a consecutive number; this is for the header on each page - perhaps a shortened title

%------------------------------------------------------------------------------
%	SECTION 1
%------------------------------------------------------------------------------

\section{Cytomics overview}

%------------------------------------------------------------------------------
%	Annotation for Microscopes
%------------------------------------------------------------------------------

\section{Annotation for Microscopes}


%-----------------------------------
%	Annotating old Data
%-----------------------------------

\subsection{Annotating old Data}

%-----------------------------------
%	Annotating new Data
%-----------------------------------

\subsection{Annotating new Data}

 
% Chapter Template

\chapter{Discussion and Conclusions} % Main chapter title

\label{Chapter6} % Change X to a consecutive number; for referencing this chapter elsewhere, use \ref{ChapterX}

\lhead{Chapter 6. \emph{Discussion and Conclusions}} % Change X to a consecutive number; this is for the header on each page - perhaps a shortened title

%------------------------------------------------------------------------------
%	Using Ontologies
%------------------------------------------------------------------------------

\section{Using Ontologies}

How can we make more use of ontologies\index{Ontologies}? What do we
miss when we do not use the reasoning\index{Reasoning} capabilities of
OWL2\index{OWL2}?

%------------------------------------------------------------------------------
%	Medical Image Sharing
%------------------------------------------------------------------------------

\section{Medical Image Sharing}

What are the deterrents for data sharing\index{Data sharing|textbf} of
medical images\index{Medical images}? We have identified several
important factors that might inhibit effective data sharing within the
medical domain. We will discuss them in the subsections Social,
Financial, Ethical and Technical.

\subsection{Social}
To examine data sharing as a behavior, we have to look into scientific
culture\index{Scientific
  culture|see{Culture,~Scientific}}\index{Culture!Scientific|textbf}. What
are the current norms and how did they come into place?  Within our
society medical scientists are so by profession. Access to medical
data in bulk is centered around hospitals\index{Hospital} and
pharmaceutical companies\index{Pharmaceutical companies}.

Within pharmaceutical companies data is an economical asset. As an
competitor on the market, it does not pay off to report your findings
to your competitors. Reporting negative results allows the competitors
to avoid spending research cost into dead-ends. Reporting positive
results only makes sense if the original company can protect the
exploitation by patents\index{Patent}.

Within hospitals, research is conducted within
academia\index{Academia}. Success is measured in terms of
publications. Only positive results typically result in high impact
publications. Funding of research is awarded more readily to successful
scientists, which means job security and success are directly related
to finding positive results. Gathering medical data is extremely labor
intensive work and is the bulk of a good research paper. If data is
shared openly, there is a possibility that other scientists can get
results with less work. These scientists will be perceived as being
more successful, because they can potentially publish more in less
time. The environment has been described as publish or
perish\index{Publish or Perish|textbf}.

Altmetrics\index{Altmetrics|textbf}\cite{Altmetrics} propose a
alternative to the traditional citation metrics. This new method seeks
to quantify the impact of scholarly work in more ways than just
citation counts, including data citations. This could change the way
data sharing is rewarded.

\subsection{Financial}

Data sharing is not free. Two important factors in data sharing are
the preservation of the data and the accessibility. At the very least,
preservation requires the duplication of the data in two different
locations and scheduled checks if the carriers of the data need to be
replaced or updated. Accessibility likewise requires maintenance to
stay up to date with the current communication methods.

\subsection{Ethical}

\subsection{Technical}
 
%\input{Chapters/Chapter7} 

%----------------------------------------------------------------------------------------
%	THESIS CONTENT - APPENDICES
%----------------------------------------------------------------------------------------

\addtocontents{toc}{\vspace{2em}} % Add a gap in the Contents, for aesthetics

\begin{appendices} % Cue to tell LaTeX that the following 'chapters' are Appendices

% Include the appendices of the thesis as separate files from the Appendices folder
% Uncomment the lines as you write the Appendices

% Appendix A

\chapter{Description Logic} % Main appendix title

\label{AppendixA} % For referencing this appendix elsewhere, use \ref{AppendixA}

\lhead{Appendix A. \emph{Description Logic}} % This is for the header on each page - perhaps a shortened title

Description Logic (DL) is based on First-Order Logic (FOL), with the following additions or modifications. DL uses slightly different nomenclature from FOL (see Table \ref{tab:nomenclature}).

\begin{table}
\centering
\begin{tabular}{|l|l|l|}
\hline
FOL & DL & OWL \\ \hline
Class & Concept & Class \\ \hline
Predicate & Role & Property \\ \hline
Object & Individual & Individual \\ \hline
\end{tabular}
\caption{Nomenclature for FOL, DL and OWL}
\label{tab:nomenclature}
\end{table}


%\input{Appendices/AppendixB}
%\input{Appendices/AppendixC}

\end{appendices}
\addtocontents{toc}{\vspace{2em}} % Add a gap in the Contents, for aesthetics

\backmatter

%----------------------------------------------------------------------------------------
%	INDEX
%----------------------------------------------------------------------------------------

\printindex

%----------------------------------------------------------------------------------------
%	BIBLIOGRAPHY
%----------------------------------------------------------------------------------------

\label{Bibliography}

\lhead{\emph{Bibliography}} % Change the page header to say "Bibliography"

\bibliographystyle{unsrtnat} % Use the "unsrtnat" BibTeX style for formatting the Bibliography

\bibliography{Bibliography} % The references (bibliography) information are stored in the file named "Bibliography.bib"

\clearpage % Start a new page

%----------------------------------------------------------------------------------------
%	ACKNOWLEDGEMENTS
%----------------------------------------------------------------------------------------

\setstretch{1.3} % Reset the line-spacing to 1.3 for body text (if it has changed)

\acknowledgements{\addtocontents{toc}{\vspace{1em}} % Add a gap in the Contents, for aesthetics

I would like to thank the Cyttron Consortium for making this research
possible. Fons Verbeek, my supervisor, for allowing me to obtain this
PhD in his group and giving me the freedom to explore at the start of
my research.

I am very grateful to my colleagues in the Imaging and Bioinformatics
group at LIACS. Yun Bei, Amalia Kallergi and Julia Dmtrieva made a
great basis to start my research. Laura Bertens has been a great help
in the parts of my research that were biological in nature. Dome
Potikanond helped when servers needed to be fixed. Special thanks to
Mohammed Tlais, Irene Martorelli, Enrique Larios, Rafael Carvalho, Lu
Cao, Di Zi, Alex Nezhinsky, Kuan Yan and Willem Davids.

In the Biology department Gerda Lamers helped me get more hands-on
experience with different types of microscopes.

Other people that have helped me at LIACS and I wish to thank are:
Floris Sicking, Robert Nagtegaal and Vianney Goovers. I am also very
grateful for all the drinks and nice conversions I had with Maarten
Lamers and Hanna Schraffenberger.

I want to thank my friends. Astronomy and physics students I started
my studies with: Jean-Paul Keulen, Rogier Ensing, Maurice Westmaas,
Sirach Franssen, Cornell Goksu and Randy Kalkman, helped me by their
comments which made me feel more confident. Thanks guys.

A thank you for the patience of other friends, which have seen very
little of me the past few years.

Last but not least, I want to thank my family. I want to thank Will
Slob and Dick Slob in particular. They have helped me tremendously by
taking a lot of non-academic work out of my hands.

Eponine and Luure, I want to thank you for being around and making my
life happier.

Sietske, you have been extremely patient and helped to motivate me
when the going was tough. A lot hapened in the time I was working on
this thesis, but we got through it together.
}

%----------------------------------------------------------------------------------------
%	CURRICULUM VITAE
%----------------------------------------------------------------------------------------

\begin{titlepage}
\chapter{Curriculum vitae}
Joris Slob was born in Oud-Beijerland, 1978. He studied Physics at
Leiden University (BSc, graduated 2004) and Media Technology at Leiden
University (MSc, graduated 2008). From 1999 to 2005, he worked as a
helpdesk consultant at RMPI. From 2005 to 2008, he worked as a
programmer at Zest Software. From 2008 to 2010, he worked as a
scientific programmer for the Imaging \& Bioinformatics group at
Leiden University. In 2010, he started his PhD research at the same
research group.
\end{titlepage}

\end{document}  

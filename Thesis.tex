%%%%%%%%%%%%%%%%%%%%%%%%%%%%%%%%%%%%%%%%%
% Masters/Doctoral Thesis 
% LaTeX Template
% Version 1.41 (9/9/13)
%
% This template has been downloaded from:
% http://www.latextemplates.com
%
% Original authors:
% Steven Gunn 
% http://users.ecs.soton.ac.uk/srg/softwaretools/document/templates/
% and
% Sunil Patel
% http://www.sunilpatel.co.uk/thesis-template/
%
% License:
% CC BY-NC-SA 3.0 (http://creativecommons.org/licenses/by-nc-sa/3.0/)
%
% Note:
% Make sure to edit document variables in the Thesis.cls file
%
%%%%%%%%%%%%%%%%%%%%%%%%%%%%%%%%%%%%%%%%%

%----------------------------------------------------------------------------------------
%	PACKAGES AND OTHER DOCUMENT CONFIGURATIONS
%----------------------------------------------------------------------------------------

\documentclass[11pt, a4paper, oneside]{Thesis} % Paper size, default font size and one-sided paper

\graphicspath{{Pictures/}} % Specifies the directory where pictures are stored

\usepackage[square, numbers, comma, sort&compress]{natbib} % Use the natbib reference package - read up on this to edit the reference style; if you want text (e.g. Smith et al., 2012) for the in-text references (instead of numbers), remove 'numbers' 
\usepackage{makeidx}
\usepackage{appendix}
\makeindex
\hypersetup{urlcolor=blue, colorlinks=true} % Colors hyperlinks in blue - change to black if annoying
\title{\ttitle} % Defines the thesis title - don't touch this

\begin{document}

\frontmatter % Use roman page numbering style (i, ii, iii, iv...) for the pre-content pages

\setstretch{1.3} % Line spacing of 1.3

% Define the page headers using the FancyHdr package and set up for one-sided printing
\fancyhead{} % Clears all page headers and footers
\rhead{\thepage} % Sets the right side header to show the page number
\lhead{} % Clears the left side page header

\pagestyle{fancy} % Finally, use the "fancy" page style to implement the FancyHdr headers

\newcommand{\HRule}{\rule{\linewidth}{0.5mm}} % New command to make the lines in the title page

% PDF meta-data
\hypersetup{pdftitle={\ttitle}}
\hypersetup{pdfsubject=\subjectname}
\hypersetup{pdfauthor=\authornames}
\hypersetup{pdfkeywords=\keywordnames}

%----------------------------------------------------------------------------------------
%	TITLE PAGE
%----------------------------------------------------------------------------------------

\begin{titlepage}
\begin{center}

{\huge \bfseries \ttitle}\\[0.4cm] % Thesis title

\textsc{\Large PROEFSCHRIFT}\\[0.5cm] % Thesis type

\large \textit{ter verkrijgen van\\
de graad van \degreename{} aan de \univname,\\
op gezag van de Rector Magnificus Prof. mr. dr. Carel Stolker,\\
volgens besluit van het College voor Promoties\\
te verdedigen op ...\\
klokke ... uur}

\large \textit{door}

\authornames

\large \textit{geboren te Oud-Beijerland op 9 mei 1978}
 
\vfill
\end{center}

\end{titlepage}

%----------------------------------------------------------------------------------------
%	PROMOTIECOMMISSIE
%----------------------------------------------------------------------------------------

\begin{titlepage}
PROMOTIECOMMISSIE

\textit{Promotor}\\
Prof. dr. J.N. Kok

\textit{Copromotor}\\
Dr. Ir. F.J. Verbeek

\textit{Overige Leden}\\
?\\
?\\
?\\
?
\vfill
This research was supported by the Cyttron project
\end{titlepage}


%----------------------------------------------------------------------------------------
%	DEDICATION
%----------------------------------------------------------------------------------------

\setstretch{1.3} % Return the line spacing back to 1.3

\pagestyle{empty} % Page style needs to be empty for this page

\dedicatory{This thesis is dedicated in loving memory of my
  parents\\ Bas and Marja. } % Dedication text

\addtocontents{toc}{\vspace{2em}} % Add a gap in the Contents, for aesthetics


%----------------------------------------------------------------------------------------
%	QUOTATION PAGE
%----------------------------------------------------------------------------------------

\pagestyle{empty} % No headers or footers for the following pages

\null\vfill % Add some space to move the quote down the page a bit

\textit{``Als je een geleerd iemand bent, moet je lang nadenken over een vraag en moet je een zwaar boek erbij pakken; dat denkt het beste."}

\begin{flushright}
Eponine de Beer (7 jaar)
\end{flushright}

\vfill\vfill\vfill\vfill\vfill\vfill\null % Add some space at the bottom to position the quote just right

\clearpage % Start a new page

%----------------------------------------------------------------------------------------
%	LIST OF CONTENTS/FIGURES/TABLES PAGES
%----------------------------------------------------------------------------------------

\pagestyle{fancy} % The page style headers have been "empty" all this time, now use the "fancy" headers as defined before to bring them back

\lhead{\emph{Contents}} % Set the left side page header to "Contents"
\tableofcontents % Write out the Table of Contents

\lhead{\emph{List of Figures}} % Set the left side page header to "List of Figures"
\listoffigures % Write out the List of Figures

\lhead{\emph{List of Tables}} % Set the left side page header to "List of Tables"
\listoftables % Write out the List of Tables

%----------------------------------------------------------------------------------------
%	ABBREVIATIONS
%----------------------------------------------------------------------------------------

\clearpage % Start a new page

\setstretch{1.5} % Set the line spacing to 1.5, this makes the following tables easier to read

\lhead{\emph{Abbreviations}} % Set the left side page header to "Abbreviations"
\listofsymbols{ll} % Include a list of Abbreviations (a table of two columns)
{
\textbf{DL} & \textbf{D}escription \textbf{L}ogic \\
\textbf{FOL} & \textbf{F}irst-\textbf{O}rder \textbf{L}ogic \\
\textbf{OWL} & \textbf{W}eb \textbf{O}ntology \textbf{L}anguage \\
\textbf{RDF} & \textbf{R}esource \textbf{D}escription \textbf{F}ramework \\
}

%----------------------------------------------------------------------------------------
%	SYMBOLS
%----------------------------------------------------------------------------------------

\clearpage % Start a new page

\lhead{\emph{Symbols}} % Set the left side page header to "Symbols"

\listofnomenclature{ll} % Include a list of Symbols (a two column table)
{
\(\top\) & Top \\ 
\(\bot\) & Bottom \\ 
\(\sqsubseteq\) & Concept Inclusion \\
\(\sqcup\) & Union of Concepts \\
\(\sqcap\) & Intersection of Concepts \\
\(\neg\) & Negation of Concepts \\
\(\forall\) & Universal Restriction \\
\(\exists\) & Existential Restriction \\
\(\equiv\) & Concept Equivalence \\
\(\doteq\) & Concept Definition \\
\(\colon\) & Concept or Role Assertion \\
% Symbol & Name \\
}

%----------------------------------------------------------------------------------------
%	THESIS CONTENT - CHAPTERS
%----------------------------------------------------------------------------------------

\mainmatter % Begin numeric (1,2,3...) page numbering

\pagestyle{fancy} % Return the page headers back to the "fancy" style

% Include the chapters of the thesis as separate files from the Chapters folder
% Uncomment the lines as you write the chapters

% Chapter Template

\chapter{Introduction} % Main chapter title

\label{Chapter1}

\lhead{Chapter 1. \emph{Introduction}} 

%------------------------------------------------------------------------------
%	Overview
%------------------------------------------------------------------------------

\section{General Overview}

The scientific method \index{Scientific method} relies on the
generation of measurements\index{Measurements}. For reproducibility
\index{Reproducibility} measurements should be reported in such a way
that other scientist can replicate the results.

Within computational science the ability to replicate results is
called Reproducible Research\index{Reproducible
  research}\cite{FomelReproducibleResearch}. For the Life
Sciences\index{Life science} this approach is more problematic,
because of the enormous amount of parameters in any given
experiment. A simple workflow\index{Workflow} might look like this:

\begin{itemize}
\item Preparation of sample\index{Sample preparation}
\item Microscopy on sample\index{Microscopy}
\item Image analysis\index{Image!Analysis}
\item Statistical analysis\index{Statistical analysis}
\end{itemize}

Each of these steps deserve a methods page on their own, yet papers
need to communicate their findings succinctly. When there is no room
for in the papers for the provenance\index{Provenance} information,
this information has to stored next to or inside the data itself.

In this thesis I will explore the questions how we can capture, store
and leverage this metadata\index{Metadata}, especially in the area of
microscopy images\index{Microscopy images}.


\subsection{Contextual information}

When talking about microscopy images\index{Microscopy images}, it is
not enough to just share the pixels\index{Pixels} of the
image. Without contextual
information\index{Contextual~information|textbf} the data is useless.

How do we measure the quality of the contextual information of a
microscope image? To get an answer, we take four steps:
\begin{itemize}
\item Analytical decomposition\index{Analytical~decomposition} of the
  pipeline\index{Pipeline} of data acquisition\index{Data acquisition}
  in microscopy science\index{Microscopy science}.
\item Survey experimental biologists, bioinformaticians, theoretical
  biologists and medical experts.
\item Check existing image repositories\index{Image!Repository} for
  their metadata\index{Metadata}.
\item Check existing metadata\index{Metadata} schemas\index{Schema}
\end{itemize}

\subsubsection{Analytical decomposition}

The Analytical decomposition
step\index{Analytical~decomposition|textbf} is performed by taking a
step backwards and providing an abstract description of the
process. For this process we use an upper
ontology\index{Upper~Ontology|see{Ontology,~Upper}}\index{Ontology!Upper}
to help guide the abstract description. Guided by an overview of upper
ontologies\cite{mascardi2007comparison}, we decided to use the
BFO\cite{grenon2004biodynamic} as our upper ontology, because of its
small size and the research focused design.
\begin{displaymath}
  \xymatrix{ & \text{Biologist} \ar[dl] \ar[d] \ar[dr] \ar@{<->}[rr] &
    & \text{Bioinformatician} \ar[dl] \ar[d] \ar[dr] &
    \\ \text{Sample} \ar[d] \ar[r] & \text{Microscope} \ar[d] \ar[r] &
    \text{Image} \ar[r] & \text{Computer} \ar[d] \ar[r] & \text{Data}
    \\ \text{Biological process} & \text{Physical interaction} & &
    \text{Computation} & }
\end{displaymath}
A sample\index{Sample} could be considered both an
object\index{Object} and an
object\_aggregate\index{Object~aggregate}. Because of this dual nature
we will model it as a material\_entity\index{Material~entity}, which
is the superclass of both. A microscope\index{Microscope} and a
computer\index{Computer} are subclasses of
object\index{Object}. Image\index{Image} and data\index{Data} are
subclass of
generically\_dependent\_continuant\index{Generically~dependent~continuant}.

Biological process, physical interaction and computation can all be
considered to be subclasses of process.

\subsubsection{Survey experts}

Survey\index{Survey|textbf} experimentalists, analysts and modelers
what the important knowledge is in their domain.

\subsubsection{Existing image repositories}

Examine the existing image repositories\index{Image!Repository} and
their choices in the metadata\index{Metadata} they include.

\subsubsection{Existing metadata schemas}

Check existing metadata schemas\index{Schema|textbf}. This includes
image formats\index{Image!Format} and ontologies\index{Ontology}.

\subsection{Assumed important elements}

I assume the following elements need to be described:
\begin{itemize}
\item Biological origin:
  Species\index{Species}/Cell culture\index{Cell culture} (including
  age), Anatomy structures\index{Anatomy structures}
\item Chemical and physical conditions\index{Chemical
  conditions}\index{Physical conditions}
\item Imaging techniques\index{Imaging techniques}
\end{itemize}

%------------------------------------------------------------------------------
%	Semantics and Reasoning
%------------------------------------------------------------------------------

\section{Semantics and Reasoning}

This section describes the main technology that is used to arrive at
new insights. What do I mean when I talk about
semantics\index{Semantics|textbf} and
reasoning\index{Reasoning|textbf}? This describes the larger field,
with a subsection delving into the semantic web\index{Semantic web}.

%-----------------------------------
%	Semantic Web
%-----------------------------------

\subsection{Semantic Web}

This section describes the Semantic Web\index{Semantic web|textbf}
idea and its main technologies. What are the restrictions, what can we
expect?

The Semantic Web is a concept that revolves around the usage of
different technologies. One of the most important languages used is
that of OWL (Web Ontology Language)\index{OWL}. In this thesis we will
be using the latest version of OWL at the time of writing, which is
OWL 2. we will use Manchester Syntax\index{Manchester Syntax} to
describe the formal structure of our
ontologies\index{Ontology}. Because we want full flexibility in our
modeling\index{Modeling}, we will use OWL 2 Full by using RDF-based
semantics\index{RDF}\index{Semantics}.



%------------------------------------------------------------------------------
%	Annotation and Retrieval
%------------------------------------------------------------------------------

\section{Annotation and Retrieval}

This section will provide information about
annotation\index{Annotation|textbf} of other resources, images in
particular. The classical use for annotation is
retrieval\index{Retrieval|textbf}. Annotation may also be used for
reasoning\index{Reasoning} and pattern discovery\index{Pattern
  discovery}.

%------------------------------------------------------------------------------
%	Life Science Images
%------------------------------------------------------------------------------

\section{Life Science Images}

Life Science Images\index{Life science images|textbf} are images that
are taken in research from the Life Sciences. These range from
brightfield microscopy\index{Brightfield~microscope} to atomic force
microscopy\index{Atomic force microscope}.

%-----------------------------------
%	Developmental biology and model organisms
%-----------------------------------

\subsection{Developmental biology and model organisms}

Developmental biology\index{Developmental biology|textbf} is the study
of developing organisms. From embryo\index{Embryo|see{Stage,
    Embryo}}\index{Stage!Embryo} to
adolescent\index{Adolescent|see{Stage,
    Adolescent}}\index{Stage!Adolescent}, the body goes through a lot
of changes. To study growth patterns and the way growth can be
disrupted by various factors, model organisms\index{Model
  organisms|textbf} are used. Examples of model organisms are: fruit
flies, zebrafish, mice and monkeys.

%-----------------------------------
%	Image Acquisition in the Life Sciences
%-----------------------------------

\subsection{Image Acquisition in the Life Sciences}

Image Acquisition\index{Image!Acquisition|textbf} in the Life Sciences
is done using specialized microscopes. From the regular brightfield
microscope\index{Brightfield~microscope} to FRAP (Fluorescence
recovery after photobleaching)\index{FRAP}.

% Chapter Template

\chapter{Image Knowledge Systems} % Main chapter title

\label{Chapter2} % Change X to a consecutive number; for referencing this chapter elsewhere, use \ref{ChapterX}

\lhead{Chapter 2. \emph{Image Knowledge Systems}} % Change X to a consecutive number; this is for the header on each page - perhaps a shortened title

%------------------------------------------------------------------------------
%	Image metadata
%------------------------------------------------------------------------------

\section{Image Metadata}

Image metadata\index{Metadata!Image|textbf} includes structural
metadata\index{Metadata!Structural} and descriptive
metadata\index{Metadata!Descriptive}. An important part of the
metadata is provenance\index{Provenance} information.

%------------------------------------------------------------------------------
%	Image Repositories
%------------------------------------------------------------------------------

\section{Image Repositories}

Image Repositories\index{Image!Repository|textbf} are places where
images can be stored and retrieved. They can differ in functionality
and range from simple FTP-servers\index{FTP-server} to full
semantic\index{Semantics} image repositories.
 
% Chapter Template

\chapter{Ontologies for Bioimages} % Main chapter title

\label{Chapter 3} % Change X to a consecutive number; for referencing this chapter elsewhere, use \ref{ChapterX}

\lhead{Chapter 3. \emph{Ontologies for Bioimages}} % Change X to a consecutive number; this is for the header on each page - perhaps a shortened title

%------------------------------------------------------------------------------
%	Imaging Ontology
%------------------------------------------------------------------------------

\section{Imaging Ontology}

Lorem ipsum dolor sit amet, consectetur adipiscing elit. Aliquam ultricies lacinia euismod. Nam tempus risus in dolor rhoncus in interdum enim tincidunt. Donec vel nunc neque. In condimentum ullamcorper quam non consequat. Fusce sagittis tempor feugiat. Fusce magna erat, molestie eu convallis ut, tempus sed arcu. Quisque molestie, ante a tincidunt ullamcorper, sapien enim dignissim lacus, in semper nibh erat lobortis purus. Integer dapibus ligula ac risus convallis pellentesque.

%------------------------------------------------------------------------------
%	Staging Ontology
%------------------------------------------------------------------------------

\section{Staging Ontology}

In animal experiments it is important to know the age of the subject. Many experiments are done in developmental stages. Absolute time measurement doesn't make sense when comparing different species or sometimes even specimens within a single species. Uncontrollable external factors can determine the speed at which an organism grows. Various staging descriptions have been developed for the varied model organisms that are used in experiments. Among them are the following staging systems:

\begin{itemize}
\item Carnegie Stages\cite{CarnegieStage} (Vertebrate Embryos)
\item Hamburger-Hamilton Stages\cite{HamburgerHamiltonStage} (Chicken Embryos)
\item Hisoaka Battle Stages\cite{HisaokaBattleStage} (Zebrafish Embryos)
\item Kimmel Stages\cite{KimmelStage} (Zebrafish Embryos)
\item Theiler Stages\cite{TheilerStage} (Mouse Embryos)
\item Yntema Stages\cite{YntemaStage} (Turtle Embryos)
\end{itemize}

Nunc posuere quam at lectus tristique eu ultrices augue venenatis. Vestibulum ante ipsum primis in faucibus orci luctus et ultrices posuere cubilia Curae; Aliquam erat volutpat. Vivamus sodales tortor eget quam adipiscing in vulputate ante ullamcorper. Sed eros ante, lacinia et sollicitudin et, aliquam sit amet augue. In hac habitasse platea dictumst.

%------------------------------------------------------------------------------
%	Biological Relative Position Ontology
%------------------------------------------------------------------------------

\section{Biological Relative Position Ontology}

Phasellus nec enim velit. Sed pellentesque tempus lectus, quis
venenatis nibh sagittis in. In vel ullamcorper lectus. Proin viverra
lectus nisl, id luctus nisl laoreet sit amet. Aliquam vel arcu
ligula. Fusce adipiscing justo et orci cursus tincidunt eget a
ligula. Curabitur in pretium sem. Mauris eget porta elit. Curabitur
commodo enim dapibus risus pulvinar, blandit rutrum mauris
gravida. Curabitur facilisis risus ipsum, at varius sem pretium
sed. Sed adipiscing luctus adipiscing. Sed id egestas urna. Curabitur
dapibus interdum nibh non interdum. Proin in ligula quis lorem aliquet
tincidunt.

% Chapter Template

\chapter{Use Case: Cyttron Database} % Main chapter title

\label{Chapter4} % Change X to a consecutive number; for referencing this chapter elsewhere, use \ref{ChapterX}

\lhead{Chapter 4. \emph{Use Case: Cyttron Database}} % Change X to a consecutive number; this is for the header on each page - perhaps a shortened title

%------------------------------------------------------------------------------
%	Cyttron overview
%------------------------------------------------------------------------------

\section{Cyttron overview}

%------------------------------------------------------------------------------
%	Annotation for Microscopes
%------------------------------------------------------------------------------

\section{Annotation for Microscopes}

%-----------------------------------
%	Annotating old data
%-----------------------------------

\subsection{Annotating old data}

%-----------------------------------
%	Annotating new data
%-----------------------------------

\subsection{Annotating new data}


 
% Chapter Template

\chapter{Use Case: Cytomics Database} % Main chapter title

\label{Chapter5} % Change X to a consecutive number; for referencing this chapter elsewhere, use \ref{ChapterX}

\lhead{Chapter 5. \emph{Cytomics Database}} % Change X to a consecutive number; this is for the header on each page - perhaps a shortened title

%------------------------------------------------------------------------------
%	SECTION 1
%------------------------------------------------------------------------------

\section{Cytomics overview}

The Cytomics database\index{Cytomics database|textbf} is a platform
that stores high-throughput\index{High-throughput} experiments.

%------------------------------------------------------------------------------
%	Annotation for Microscopes
%------------------------------------------------------------------------------

\section{Annotation for Microscopes}

Because of every lab only has a limited amount of
microscopes\index{Microscope}, every microscope can be well
annotated\index{Annotation} and the metadata\index{Metadata} can be
attached to the images it generates.

%-----------------------------------
%	Annotating old Data
%-----------------------------------

\subsection{Annotating old Data}

Before the Imaging Ontology\index{Imaging ontology} was made, the
Cytomics database\index{Cytomics database} already accepted entries. We therefore have to annotate old data\index{Annotating!old data}.

%-----------------------------------
%	Annotating new Data
%-----------------------------------

\subsection{Annotating new Data}

New data that enters the database can be automatically enhanced with
appropriate metadata\index{Metadata}. The
annotation\index{Annotating!new data} of new data is done using
predefined imaging equipment and image header\index{Image!Header}
parsing\index{Parsing}.
 
% Chapter Template

\chapter{Discussion and Conclusions} % Main chapter title

\label{Chapter6} % Change X to a consecutive number; for referencing this chapter elsewhere, use \ref{ChapterX}

\lhead{Chapter 6. \emph{Discussion and Conclusions}} % Change X to a consecutive number; this is for the header on each page - perhaps a shortened title

%------------------------------------------------------------------------------
%	Using Ontologies
%------------------------------------------------------------------------------

\section{Using Ontologies}

%------------------------------------------------------------------------------
%	Medical Image Sharing
%------------------------------------------------------------------------------

\section{Medical Image Sharing}

 
%\input{Chapters/Chapter7} 

%----------------------------------------------------------------------------------------
%	THESIS CONTENT - APPENDICES
%----------------------------------------------------------------------------------------

\addtocontents{toc}{\vspace{2em}} % Add a gap in the Contents, for aesthetics

\begin{appendices} % Cue to tell LaTeX that the following 'chapters' are Appendices

% Include the appendices of the thesis as separate files from the Appendices folder
% Uncomment the lines as you write the Appendices

% Appendix A

\chapter{Description Logic} % Main appendix title

\label{AppendixA} % For referencing this appendix elsewhere, use \ref{AppendixA}

\lhead{Appendix A. \emph{Description Logic}} % This is for the header on each page - perhaps a shortened title

Description Logic (DL) is based on First-Order Logic (FOL), with the following additions or modifications. DL uses slightly different nomenclature from FOL (see Table \ref{tab:nomenclature}).

\begin{table}
\centering
\begin{tabular}{|l|l|l|}
\hline
FOL & DL & OWL \\ \hline
Class & Concept & Class \\ \hline
Predicate & Role & Property \\ \hline
Object & Individual & Individual \\ \hline
\end{tabular}
\caption{Nomenclature for FOL, DL and OWL}
\label{tab:nomenclature}
\end{table}


%\input{Appendices/AppendixB}
%\input{Appendices/AppendixC}

\end{appendices}
\addtocontents{toc}{\vspace{2em}} % Add a gap in the Contents, for aesthetics

\backmatter

%----------------------------------------------------------------------------------------
%	INDEX
%----------------------------------------------------------------------------------------

\printindex

%----------------------------------------------------------------------------------------
%	BIBLIOGRAPHY
%----------------------------------------------------------------------------------------

\label{Bibliography}

\lhead{\emph{Bibliography}} % Change the page header to say "Bibliography"

\bibliographystyle{unsrtnat} % Use the "unsrtnat" BibTeX style for formatting the Bibliography

\bibliography{Bibliography} % The references (bibliography) information are stored in the file named "Bibliography.bib"

\clearpage % Start a new page

%----------------------------------------------------------------------------------------
%	ACKNOWLEDGEMENTS
%----------------------------------------------------------------------------------------

\setstretch{1.3} % Reset the line-spacing to 1.3 for body text (if it has changed)

\acknowledgements{\addtocontents{toc}{\vspace{1em}} % Add a gap in the Contents, for aesthetics

I would like to thank the Cyttron Consortium for making this research
possible. Fons Verbeek, my supervisor, for allowing me to obtain this
PhD in his group and giving me the freedom to explore at the start of
my research.

I am very grateful to my colleagues in the Imaging and Bioinformatics
group at LIACS. Yun Bei, Amalia Kallergi and Julia Dmtrieva made a
great basis to start my research. Laura Bertens has been a great help
in the parts of my research that were biological in nature. Dome
Potikanond helped when servers needed to be fixed. Special thanks to
Mohammed Tlais, Irene Martorelli, Enrique Larios, Rafael Carvalho, Lu
Cao, Di Zi, Alex Nezhinsky, Kuan Yan and Willem Davids.

In the Biology department Gerda Lamers helped me get more hands-on
experience with different types of microscopes.

Other people that have helped me at LIACS and I wish to thank are:
Floris Sicking, Robert Nagtegaal and Vianney Goovers. I am also very
grateful for all the drinks and nice conversions I had with Maarten
Lamers and Hanna Schraffenberger.

I want to thank my friends. Astronomy and physics students I started
my studies with: Jean-Paul Keulen, Rogier Ensing, Maurice Westmaas,
Sirach Franssen, Cornell Goksu and Randy Kalkman, helped me by their
comments which made me feel more confident. Thanks guys.

A thank you for the patience of other friends, which have seen very
little of me the past few years.

Last but not least, I want to thank my family. I want to thank Will
Slob and Dick Slob in particular. They have helped me tremendously by
taking a lot of non-academic work out of my hands.

Eponine and Luure, I want to thank you for being around and making my
life happier.

Sietske, you have been extremely patient and helped to motivate me
when the going was tough. A lot hapened in the time I was working on
this thesis, but we got through it together.
}

%----------------------------------------------------------------------------------------
%	CURRICULUM VITAE
%----------------------------------------------------------------------------------------

\begin{titlepage}
\chapter{Curriculum vitae}
Joris Slob was born in Oud-Beijerland, 1978. He studied Physics at
Leiden University (BSc, graduated 2004) and Media Technology at Leiden
University (MSc, graduated 2008). From 1999 to 2005, he worked as a
helpdesk consultant at RMPI. From 2005 to 2008, he worked as a
programmer at Zest Software. From 2008 to 2010, he worked as a
scientific programmer for the Imaging \& Bioinformatics group at
Leiden University. In 2010, he started his PhD research at the same
research group.
\end{titlepage}

\end{document}  

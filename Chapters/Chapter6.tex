% Chapter Template

\chapter{Discussion and Conclusions} % Main chapter title

\label{Chapter6} % Change X to a consecutive number; for referencing this chapter elsewhere, use \ref{ChapterX}

\lhead{Chapter 6. \emph{Discussion and Conclusions}} % Change X to a consecutive number; this is for the header on each page - perhaps a shortened title

%------------------------------------------------------------------------------
%	Using Ontologies
%------------------------------------------------------------------------------

\section{Using Ontologies}

How can we make more use of ontologies\index{Ontologies}? What do we
miss when we do not use the reasoning\index{Reasoning} capabilities of
OWL2\index{OWL2}?

%------------------------------------------------------------------------------
%	Medical Image Sharing
%------------------------------------------------------------------------------

\section{Medical Image Sharing}

What are the deterrents for data sharing\index{Data sharing|textbf} of
medical images\index{Medical images}? We have identified several
important factors that might inhibit effective data sharing within the
medical domain. We will discuss them in the subsections Social,
Financial, Ethical and Technical.

\subsection{Social}
To examine data sharing as a behavior, we have to look into scientific
culture\index{Scientific
  culture|see{Culture,~Scientific}}\index{Culture!Scientific|textbf}. What
are the current norms and how did they come into place?  Within our
society medical scientists are so by profession. Access to medical
data in bulk is centered around hospitals\index{Hospital} and
pharmaceutical companies\index{Pharmaceutical companies}.

Within pharmaceutical companies data is an economical asset. As an
competitor on the market, it does not pay off to report your findings
to your competitors. Reporting negative results allows the competitors
to avoid spending research cost into dead-ends. Reporting positive
results only makes sense if the original company can protect the
exploitation by patents\index{Patent}.

Within hospitals, research is conducted within
academia\index{Academia}. Success is measured in terms of
publications. Only positive results typically result in high impact
publications. Funding of research is awarded more readily to successful
scientists, which means job security and success are directly related
to finding positive results. Gathering medical data is extremely labor
intensive work and is the bulk of a good research paper. If data is
shared openly, there is a possibility that other scientists can get
results with less work. These scientists will be perceived as being
more successful, because they can potentially publish more in less
time. The environment has been described as publish or
perish\index{Publish or Perish|textbf}.

Altmetrics\index{Altmetrics|textbf}\cite{Altmetrics} propose a
alternative to the traditional citation metrics. This new method seeks
to quantify the impact of scholarly work in more ways than just
citation counts, including data citations. This could change the way
data sharing is rewarded.

\subsection{Financial}

Data sharing is not free. Two important factors in data sharing are
the preservation of the data and the accessibility. At the very least,
preservation requires the duplication of the data in two different
locations and scheduled checks if the carriers of the data need to be
replaced or updated. Accessibility likewise requires maintenance to
stay up to date with the current communication methods.

\subsection{Ethical}

Data sharing within the medical domain is a tricky subject. Patient
privacy is considered extremely important and might even prevail over
the life saving potential that data sharing has to offer. With correct
patient consent forms some form of data sharing is possible, but there
still are many grey areas, like the case of withdrawal of consent and
consent from parents for under-aged children. Anonymization cannot
offer a solution in all cases, because database coupling sometimes
leads to unintentional de-anonymization. Personal disclosure of
medical information might lead to potential de-anonymization of other
patients, because images can be analyzed and specific machine
fingerprints together with other information can lead to very small
sets of possible patients the data can belong to.

On the other hand, data hiding is a problem for the scientific
method. Some effects can only be measured using large samples, which
are unavailable for some rare conditions. This makes it virtually
impossible for certain experiments to be verified by repetition. If
the raw data is also unavailable because of privacy concerns, the
basis of scientific conclusions becomes unscientific.

Another issue in the ethical domain is the question of ownership. The
acquired data can potentially belong to many parties. Firstly, most
medical research images are made using research grants from public
money. It is possible to argue that the results of this research
should therefore be made available to the public.

The researchers designing the experiment and putting all the effort
into acquiring the images may rightfully feel entitled to call the
data theirs. In extension, the institute they work for can claim
ownership of the data.

The subject undergoing the experiment can argue that since the image
contains personal information, the image ownership should be theirs
and they can grant permission for the image to be used and revoke that
right at any time.

\subsection{Technical}

Data sharing has become a lot easier with the rise of the Internet and
cloud services. Storage space is becoming cheaper and cheaper, but on
the other hand, the amount of experimental data is also increasing
drastically. This has resulted in the possibility to share research
data with effort, but it is not an off the shelf possibility.

Apart from the size of data storage, the data is also stored in a
particular format. These formats can become deprecated and over time
the software that originally opened the files might be obsolete and no
longer available. This means that data sharing over a longer period of
time also encompasses data stewardship. Data stewardship includes the
obligation to migrate the data to newer formats or to keep alive the
technology to read the data. It is difficult to estimate the cost of
stewardship, but virtual machines have provided a technical solution
that mitigates the problem somewhat.

% Chapter Template

\chapter{Discussion and Conclusions} % Main chapter title

\label{Chapter6} % Change X to a consecutive number; for referencing this chapter elsewhere, use \ref{ChapterX}

\lhead{Chapter 6. \emph{Discussion and Conclusions}} % Change X to a consecutive number; this is for the header on each page - perhaps a shortened title

%------------------------------------------------------------------------------
%	Using Ontologies
%------------------------------------------------------------------------------

\section{Using Ontologies}

How can we make more use of ontologies\index{Ontologies}? What do we
miss when we do not use the reasoning\index{Reasoning} capabilities of
OWL2\index{OWL2}?

%------------------------------------------------------------------------------
%	Medical Image Sharing
%------------------------------------------------------------------------------

\section{Medical Image Sharing}

What are the deterents for data sharing\index{Data sharing|textbf} of
medical images\index{Medical images}? We have identified several
important factors that might inhibit effective data sharing within the
medical domain. We will discuss them in the subsections Social,
Financial, Ethical and Technical.

\subsection{Social}
To examine data sharing as a behavior, we have to look into scientific
culture\index{Scientific
  culture|see{Culture,~Scientific}}\index{Culture!Scientific|textbf}. What
are the current norms and how did they come into place?  Within our
society medical scientists are so by profession. Access to medical
data is centered around hospitals\index{Hospital} and pharmaceutical
companies\index{Pharmaceutical companies}.

Within pharmaceutical companies data is an economical asset. As an
competitor on the market, it does not pay off to report your findings
to your competitors. Reporting negative results allows the competitors
to avoid spending research cost into dead-ends. Reporting positive
results only makes sense if the original company can protect the
exploitation by patents\index{Patent}.

Within hospitals, research is conducted within
academia\index{Academia}. Success is measured in terms of
publications. Only positive results typically result in high impact
publications. Funding of research is awarded more readily to succesful
scientists, which means job security and success are directly related
to finding positive results. Gathering medical data is extremely labor
intensive work and is the bulk of a good research paper. If data is
shared openly, there is a possibility that other scientists can get
results with less work. These scientists will be perceived as being
more successful, because they can potentially publish more in less
time. The environment has been described as publish or
perish\index{Publish or Perish|textbf}.

Altmetrics\index{Altmetrics|textbf} propose a alternative to the
traditional citation metrics. This new method seeks to quantify the
impact of scholarly work in more ways than just citation counts.

\subsection{Financial}

\subsection{Ethical}

\subsection{Technical}

% Chapter Template

\chapter{Ontologies for Bioimages} % Main chapter title

\label{Chapter 3} % Change X to a consecutive number; for referencing this chapter elsewhere, use \ref{ChapterX}

\lhead{Chapter 3. \emph{Ontologies for Bioimages}} % Change X to a consecutive number; this is for the header on each page - perhaps a shortened title

%------------------------------------------------------------------------------
%	Imaging Ontology
%------------------------------------------------------------------------------

\section{Imaging Ontology}

The microscopy lab at Leiden University both contains commercial
standardized microscopes and specialized custom built
microscopes. Experimental biologists need the flexibility to customize
their equipment to find new ways to test their hypotheses. The images
that were acquired are often sent to image analysts to gain statistics
from their measurements. The image analysis often requires a good
understanding of the exact methods by which the images were
acquired. Moreover, when sharing the data after publication, it is
important to have correct provenance information.

A tried solution has been the creation of imaging ontologies. These
ontologies were mostly created by single labs which focus on their
particular use-cases. Although most of the labs take care to abstract
and broaden the scope of their particular ontology, this model will
not scale. In practice any additions that an outside lab wants to make
to the ontology will be reviewed with the goals of the original lab in
mind.

We propose an client-server architecture for our imaging ontology. In
this architecture, there will be a central ontology, called the master
ontology, that holds the biggest subset of statements that all the
labs can agree on. Every lab will have their own local imaging
ontology that extends the master ontology.

Local ontologies will be placed inside their own local namespace. This
ensures that no name clashes will occur between different local
ontologies. Images are always annotated according to a local
ontology. When an image is shared, the local classes are transformed
into their most specific master ontology parent class. This allows for
the most semantically relevant information to be shared, while keeping
any secret microscope information hidden.

The extension of the local ontologies will be driven by a user
interface that limits the addition of rules that are consistent with
the current master and local ontology. Statements from all the local
ontologies are aggregated and inspected by domain and ontology experts
who will update the master ontology.

The update process proceeds along the following steps:
\begin{itemize}
  \item A pattern is observed from local ontologies
  \item A domain expert describes the changes that have to be made
  \item An ontology expert constructs the rules to transform the ontologies
  \item The transformation is tested on the master and local ontologies and validated by reasoners
  \item Any validation errors are resolved by the ontology expert or the change is postponed until consensus is reached
  \item The validated changes are put into place
\end{itemize}

The technique used here is best described by the software pattern
called the observer pattern. Using the terminology of the observer
pattern, the master ontology is the subject and the local ontologies
are observers. The local ontologies are notified when the master
ontology requires local changes.

The transformations on the master and local ontologies are sent using
SPARQL\index{SPARQL} 1.1 constructs. In this version of SPARQL,
inserts and deletions are possible. In our initial setup the local
ontologies are located on the same system as the master ontology, but
nothing in this architecture requires the local ontology to be on the
same system.

To start this updating process we have to construct a reasonable first
master version of the imaging ontology. The imaging
ontology\index{Imaging~ontology|see{Ontology,~Imaging}}\index{Ontology!Imaging|textbf}
describes imaging equipment\index{Imaging equipment} and their related
modalities\index{Modality}. It builds on PROV\index{PROV} and includes
semantics\index{Semantics} to facilitate reasoning\index{Reasoning}
over microscope types.



%------------------------------------------------------------------------------
%	Staging Ontology
%------------------------------------------------------------------------------

\section{Staging Ontology}

In animal experiments\index{Animal experiments} it is important to
know the age of the subject. Many experiments are done in
developmental stages\index{Developmental stages}. Absolute time
measurement doesn't make sense when comparing different species or
sometimes even specimens within a single species. Uncontrollable
external factors can determine the speed at which an organism
grows. Various staging descriptions have been developed for the varied
model organisms that are used in experiments. Among them are the
following staging systems:

\begin{itemize}
\item Carnegie Stages\cite{CarnegieStage}\index{Stages!Carnegie}
  (Vertebrate Embryos)
\item Hamburger-Hamilton
  Stages\cite{HamburgerHamiltonStage}\index{Stages!Hamburger-Hamilton}
  (Chicken Embryos)
\item Hisoaka Battle
  Stages\cite{HisaokaBattleStage}\index{Stages!Hisoaka Battle}
  (Zebrafish Embryos)
\item Kimmel Stages\cite{KimmelStage}\index{Stages!Kimmel} (Zebrafish
  Embryos)
\item Theiler Stages\cite{TheilerStage}\index{Stages!Theiler} (Mouse
  Embryos)
\item Yntema Stages\cite{YntemaStage}\index{Stages!Yntema} (Turtle
  Embryos)
\end{itemize}

All these stages are defined in the Staging
Ontology\index{Staging~Ontology|see{Ontology,~Staging}}\index{Ontology!Staging|textbf}. Although
time has been modelled in the past in OWL, the main domain was regular
date and clock time. In the biological domain, we see other patterns
to describe the temporal state of living organisms. In embryology
staging schemes are used to describe the development of a
subject. These staging schemes usually rely on visual characteristics
of the subject to determine the progression along the maturation
process. To capture this sense of time, we developed the staging
ontology, which combines several staging schemes together, expressed
in OWL and quantifies them using properties and creates parallels
using relations.


%------------------------------------------------------------------------------
%	Biological Relative Position Ontology
%------------------------------------------------------------------------------

\section{Biological Relative Position Ontology}

The Biological Relative Position
Ontology\index{Biological~Relative~Position~Ontology|see{Ontology,~Biological~Relative~Position}}\index{Ontology!Biological~Relative~Position|textbf}
is an ontology\index{Ontology} that describes positions between
entities. These spatial relations have been described in the [TODO:
  look up reference]. We used this ontology and show that it can be
used to characterize and reason over spatial data. In our use case we
made use of TDR-3D to create 3D models from microscopy images. These
models are then parsed and automatically converted into a logical
spatial model. This model can then be used for validation and
querying.

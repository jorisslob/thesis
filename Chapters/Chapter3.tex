% Chapter Template

\chapter{Ontologies for Bioimages} % Main chapter title

\label{Chapter 3} % Change X to a consecutive number; for referencing this chapter elsewhere, use \ref{ChapterX}

\lhead{Chapter 3. \emph{Ontologies for Bioimages}} % Change X to a consecutive number; this is for the header on each page - perhaps a shortened title

%------------------------------------------------------------------------------
%	Imaging Ontology
%------------------------------------------------------------------------------

\section{Imaging Ontology}

Lorem ipsum dolor sit amet, consectetur adipiscing elit. Aliquam ultricies lacinia euismod. Nam tempus risus in dolor rhoncus in interdum enim tincidunt. Donec vel nunc neque. In condimentum ullamcorper quam non consequat. Fusce sagittis tempor feugiat. Fusce magna erat, molestie eu convallis ut, tempus sed arcu. Quisque molestie, ante a tincidunt ullamcorper, sapien enim dignissim lacus, in semper nibh erat lobortis purus. Integer dapibus ligula ac risus convallis pellentesque.

%------------------------------------------------------------------------------
%	Staging Ontology
%------------------------------------------------------------------------------

\section{Staging Ontology}

In animal experiments it is important to know the age of the subject. Many experiments are done in developmental stages. Absolute time measurement doesn't make sense when comparing different species or sometimes even specimens within a single species. Uncontrollable external factors can determine the speed at which an organism grows. Various staging descriptions have been developed for the varied model organisms that are used in experiments. Among them are the following staging systems:

\begin{itemize}
\item Carnegie Stages\cite{CarnegieStage} (Vertebrate Embryos)
\item Hamburger-Hamilton Stages\cite{HamburgerHamiltonStage} (Chicken Embryos)
\item Hisoaka Battle Stages\cite{HisaokaBattleStage} (Zebrafish Embryos)
\item Kimmel Stages\cite{KimmelStage} (Zebrafish Embryos)
\item Theiler Stages\cite{TheilerStage} (Mouse Embryos)
\item Yntema Stages\cite{YntemaStage} (Turtle Embryos)
\end{itemize}

Nunc posuere quam at lectus tristique eu ultrices augue venenatis. Vestibulum ante ipsum primis in faucibus orci luctus et ultrices posuere cubilia Curae; Aliquam erat volutpat. Vivamus sodales tortor eget quam adipiscing in vulputate ante ullamcorper. Sed eros ante, lacinia et sollicitudin et, aliquam sit amet augue. In hac habitasse platea dictumst.

%------------------------------------------------------------------------------
%	Biological Relative Position Ontology
%------------------------------------------------------------------------------

\section{Biological Relative Position Ontology}

Phasellus nec enim velit. Sed pellentesque tempus lectus, quis
venenatis nibh sagittis in. In vel ullamcorper lectus. Proin viverra
lectus nisl, id luctus nisl laoreet sit amet. Aliquam vel arcu
ligula. Fusce adipiscing justo et orci cursus tincidunt eget a
ligula. Curabitur in pretium sem. Mauris eget porta elit. Curabitur
commodo enim dapibus risus pulvinar, blandit rutrum mauris
gravida. Curabitur facilisis risus ipsum, at varius sem pretium
sed. Sed adipiscing luctus adipiscing. Sed id egestas urna. Curabitur
dapibus interdum nibh non interdum. Proin in ligula quis lorem aliquet
tincidunt.

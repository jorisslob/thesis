% Chapter Template

\chapter{Use Case: Cyttron Database} % Main chapter title

\label{Chapter4} % Change X to a consecutive number; for referencing this chapter elsewhere, use \ref{ChapterX}

\lhead{Chapter 4. \emph{Use Case: Cyttron Database}} % Change X to a consecutive number; this is for the header on each page - perhaps a shortened title

%------------------------------------------------------------------------------
%	Cyttron overview
%------------------------------------------------------------------------------

\section{Cyttron overview}
The Cyttron Database\index{Cyttron database|textbf} is an image
database that uses classes from ontologies to annotate microscopy
images from the life sciences. When a user uploads an image to the
database, this user is guided through a user interface that asks for
classes to annotate this image and provide additional administrative
metadata. The ontology classes form the descriptive part of the
metadata. Most of the structural metadata can be extracted from the
header of the image file that was uploaded.

The Cyttron database covers a very broad subject area: the life
sciences. This is much broader than comparable online databases [TODO:
  add reference to cell centered image database and others]. No single
ontology describes all the different views that our users might want
to express. That is why the Cyttron database uses as many ontologies
(first from OBO Foundry, now from BioPortal) as possible. Although the
possibility of annotation is a lot larger, it becomes harder for the
user to find the precise class he wants to use.

To overcome the search problem, we have devised several user
interfaces to help the user. Central in this assistance is the
possibility for users to collect terms they use often in a
personalized list called 'my terms'. Researchers often work for
extended periods on a single subject and collect many images for the
same purpose. When actually annotating images the user can select the
terms from the 'my terms' list.

To populate the my term list, we provide simple string searching
possibilities augmented with graph visualizations of ontologies (like
the ontology viewer [TODO: add reference]) to better understand the
context of the found classes and to find even better matches.

%------------------------------------------------------------------------------
%	Annotation for Microscopes
%------------------------------------------------------------------------------

\section{Annotation for Microscopes}

In my research I have explored how we can augment the data we have on
the creation of the images in Cyttron\index{Cyttron database}. During
the
acquisition\index{Acquisition|see{Image,~Acquisition}}\index{Image!Acquisition}
of images, metadata\index{Metadata} needs to be gathered and
translated into a common integrated schema\index{Schema}. The
microscope\index{Microscope} that was used for the acquisition plays
an important role when trying to understand the meaning of the image
data. The imaging ontology was used in the Cyttron database to
integrate the images. Because this ontology was introduced later,
there was legacy data that was described using conventional relational
database tables. We will describe how we migrated the old data to the
new imaging ontology and how new data is now entered into the system.

%-----------------------------------
%	Annotating old data
%-----------------------------------

\subsection{Annotating old data}

In this section we explore how old data can be retroactively be
annotated\index{Annotating!old data}.

%-----------------------------------
%	Annotating new data
%-----------------------------------

\subsection{Annotating new data}

In this section we explore how new data can be most effectively be
annotated\index{Annotating!new data}.

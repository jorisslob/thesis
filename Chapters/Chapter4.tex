% Chapter Template

\chapter{Use Case: Cyttron Database} % Main chapter title

\label{Chapter4} % Change X to a consecutive number; for referencing this chapter elsewhere, use \ref{ChapterX}

\lhead{Chapter 4. \emph{Use Case: Cyttron Database}} % Change X to a consecutive number; this is for the header on each page - perhaps a shortened title

%------------------------------------------------------------------------------
%	Cyttron overview
%------------------------------------------------------------------------------

\section{Cyttron overview}
The Cyttron Database\index{Cyttron database} is an image database that uses
semantic keywords to annotate images.

%------------------------------------------------------------------------------
%	Annotation for Microscopes
%------------------------------------------------------------------------------

\section{Annotation for Microscopes}

In my research I have explored how we can augment the data we have on
the creation on the images in Cyttron\index{Cyttron database}. During
the
acquisition\index{Aquisition|see{Image,~Acquisition}}\index{Image!Acquisition}
of images, metadata\index{Metadata} needs to be gathered and
translated into a common integrated schema\index{Schema}. The
microscope\index{Microscope} that was used for the acquisition plays
an important role when trying to understand the meaning of the image
data.

%-----------------------------------
%	Annotating old data
%-----------------------------------

\subsection{Annotating old data}

In this section we explore how old data can be retroactively be
annotated\index{Annotating!old data}.

%-----------------------------------
%	Annotating new data
%-----------------------------------

\subsection{Annotating new data}

In this section we explore how new data can be most effectively be
annotated\index{Annotating!new data}.

% Chapter Template

\chapter{Introduction} % Main chapter title

\label{Chapter1} % Change X to a consecutive number; for referencing this chapter elsewhere, use \ref{ChapterX}

\lhead{Chapter 1. \emph{Introduction}} % Change X to a consecutive number; this is for the header on each page - perhaps a shortened title

%------------------------------------------------------------------------------
%	Overview
%------------------------------------------------------------------------------

\section{General Overview}

Will semantic metadata \index{metadata} improve sharing, verifying and repurposing microscopy images from the life sciences?

\subsection{Contextual information}

When talking about microscopy images, it is not enough to just share the pixels of the image. Without contextual information the data is useless.

How do we measure the quality of the contextual information of a microscope image? To get an answer, we take three steps:
\begin{itemize}
\item Analytical decomposition of the pipeline of data acquisition in microscopy science.
\item Survey experimental biologists, bio-informaticians, theoretical biologists and medical experts.
\item Check existing image repositories for their metadata.
\end{itemize}

I assume the following elements need to be described:
\begin{itemize}
\item Biological origin: Species/Cell culture (including age), Anatomy structures
\item Chemical and physical conditions
\item Imaging techniques
\end{itemize}

%------------------------------------------------------------------------------
%	Semantics and Reasoning
%------------------------------------------------------------------------------

\section{Semantics and Reasoning}

This section describes the main technology that is used to arive at new insights. What do I mean when I talk about semantics and reasoning? This describes the larger field, with a subsection delving into the semantic web.

%-----------------------------------
%	Semantic Web
%-----------------------------------

\subsection{Semantic Web}

This section describes the Semantic Web idea and its main technologies. What are the restrictions, what can we expect?

The Semantic Web is a concept that revolves around the usage of
different technologies. One of the most important languages used is
that of OWL (Web Ontology Language). In this thesis we will be using
the latest version of OWL at the time of writing, which is OWL 2. we
will use Machester Syntax to describe the formal structure of our
ontologies. Because we want full flexibility in our modeling, we will
use OWL 2 Full by using RDF-based semantics.



%------------------------------------------------------------------------------
%	Annotation and Retrieval
%------------------------------------------------------------------------------

\section{Annotation and Retrieval}

This section will provide information about annotation of other resources, images in particular. The classical use for annotation is retrieval. Annotation may also be used for reasoning and pattern discovery.

%------------------------------------------------------------------------------
%	Life Science Images
%------------------------------------------------------------------------------

\section{Life Science Images}

Cras venenatis elit sed erat ultrices, at commodo nunc
interdum. Aenean mattis ante nec orci varius tempus. Aenean augue est,
adipiscing eget ligula sit amet, rutrum molestie augue. Duis laoreet
sem ac ligula accumsan, et convallis orci lobortis. Vivamus quis leo
varius, consequat tortor nec, viverra libero. Duis aliquet mauris eget
volutpat blandit. Nam vel aliquet neque. Nam in nibh sed elit porta
hendrerit ac non est. Phasellus vel tellus accumsan lacus porta
imperdiet. Aenean at augue libero. Aliquam pharetra nunc ut tortor
lobortis vehicula. Aenean vestibulum nec neque a ullamcorper. Proin
tristique orci facilisis, cursus dui ac, euismod neque. Sed euismod
convallis quam id sagittis. Donec egestas erat turpis.

%-----------------------------------
%	Developmental biology and model organisms
%-----------------------------------

\subsection{Developmental biology and model organisms}

Morbi rutrum odio eget arcu adipiscing sodales. Aenean et purus a est
pulvinar pellentesque. Cras in elit neque, quis varius elit. Phasellus
fringilla, nibh eu tempus venenatis, dolor elit posuere quam, quis
adipiscing urna leo nec orci. Sed nec nulla auctor odio aliquet
consequat. Ut nec nulla in ante ullamcorper aliquam at sed
dolor. Phasellus fermentum magna in augue gravida cursus. Cras sed
pretium lorem. Pellentesque eget ornare odio. Proin accumsan, massa
viverra cursus pharetra, ipsum nisi lobortis velit, a malesuada dolor
lorem eu neque.

%-----------------------------------
%	Image Acquisition in the Life Sciences
%-----------------------------------

\subsection{Image Acquisition in the Life Sciences}

Quisque luctus ullamcorper massa, at adipiscing nibh condimentum
sed. Nulla bibendum, tellus id iaculis rhoncus, eros velit auctor
turpis, a semper ipsum dui at augue. Vivamus dictum tellus urna, et
sagittis lectus cursus et. Nam ac vulputate turpis, in pellentesque
odio. Donec tristique commodo feugiat. Cras posuere, turpis molestie
malesuada malesuada, sapien sem aliquam lorem, sed malesuada diam nisi
ut orci. Maecenas et est lacus. Phasellus lacinia odio erat. Phasellus
tellus lectus, blandit luctus tristique vitae, suscipit eget
tortor. Quisque quis ipsum in velit posuere viverra. Cras sed pulvinar
diam. Etiam egestas adipiscing scelerisque.

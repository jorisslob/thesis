% Chapter Template

\chapter{Introduction} % Main chapter title

\label{Chapter1}

\lhead{Chapter 1. \emph{Introduction}} 

%------------------------------------------------------------------------------
%	Overview
%------------------------------------------------------------------------------

\section{General Overview}

The scientific method \index{Scientific method} relies on the
generation of measurements \index{Measurements}. For reproducability
\index{Reproducability} measurements should be reported in such a way
that other scientist can replicate the results.

Reproducable Research (Sergey Fomel and Jon Claerbout, Guest Editor's
Introduction: Reproducable Research doi:10.1109/MCSE.2009.14)
\index{Reproducable Research}.

Will semantic metadata improve sharing, verifying and repurposing
microscopy images from the life sciences?

\subsection{Contextual information}

When talking about microscopy images, it is not enough to just share
the pixels of the image. Without contextual information the data is
useless.

How do we measure the quality of the contextual information of a
microscope image? To get an answer, we take three steps:
\begin{itemize}
\item Analytical decomposition of the pipeline of data acquisition in
  microscopy science.
\item Survey experimental biologists, bio-informaticians, theoretical
  biologists and medical experts.
\item Check existing image repositories for their metadata.
\end{itemize}

I assume the following elements need to be described:
\begin{itemize}
\item Biological origin: Species/Cell culture (including age), Anatomy
  structures
\item Chemical and physical conditions
\item Imaging techniques
\end{itemize}

%------------------------------------------------------------------------------
%	Semantics and Reasoning
%------------------------------------------------------------------------------

\section{Semantics and Reasoning}

This section describes the main technology that is used to arive at
new insights. What do I mean when I talk about semantics and
reasoning? This describes the larger field, with a subsection delving
into the semantic web.

%-----------------------------------
%	Semantic Web
%-----------------------------------

\subsection{Semantic Web}

This section describes the Semantic Web idea and its main
technologies. What are the restrictions, what can we expect?

The Semantic Web is a concept that revolves around the usage of
different technologies. One of the most important languages used is
that of OWL (Web Ontology Language). In this thesis we will be using
the latest version of OWL at the time of writing, which is OWL 2. we
will use Machester Syntax to describe the formal structure of our
ontologies. Because we want full flexibility in our modeling, we will
use OWL 2 Full by using RDF-based semantics.



%------------------------------------------------------------------------------
%	Annotation and Retrieval
%------------------------------------------------------------------------------

\section{Annotation and Retrieval}

This section will provide information about annotation of other
resources, images in particular. The classical use for annotation is
retrieval. Annotation may also be used for reasoning and pattern
discovery.

%------------------------------------------------------------------------------
%	Life Science Images
%------------------------------------------------------------------------------

\section{Life Science Images}


%-----------------------------------
%	Developmental biology and model organisms
%-----------------------------------

\subsection{Developmental biology and model organisms}


%-----------------------------------
%	Image Acquisition in the Life Sciences
%-----------------------------------

\subsection{Image Acquisition in the Life Sciences}



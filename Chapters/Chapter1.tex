% Chapter Template

\chapter{Introduction} % Main chapter title

\label{Chapter1}

\lhead{Chapter 1. \emph{Introduction}} 

%------------------------------------------------------------------------------
%	Overview
%------------------------------------------------------------------------------

\section{General Overview}

The scientific method \index{Scientific method} relies on the
generation of measurements\index{Measurements}. For reproducibility
\index{Reproducibility} measurements should be reported in such a way
that other scientist can replicate the results.

Within computational science the ability to replicate results is
called Reproducible Research\index{Reproducible
  research}\cite{FomelReproducibleResearch}. For the Life
Sciences\index{Life science} this approach is more problematic,
because of the enormous amount of parameters in any given
experiment. A simple workflow\index{Workflow} might look like this:

\begin{itemize}
\item Preparation of sample\index{Sample preparation}
\item Microscopy on sample\index{Microscopy}
\item Image analysis\index{Image!Analysis}
\item Statistical analysis\index{Statistical analysis}
\end{itemize}

Each of these steps deserve a methods page on their own, yet papers
need to communicate their findings succinctly. When there is no room
for in the papers for the provenance\index{Provenance} information,
this information has to stored next to or inside the data itself.

In this thesis I will explore the questions how we can capture, store
and leverage this metadata\index{Metadata}, especially in the area of
microscopy images\index{Microscopy images}.


\subsection{Contextual information}

When talking about microscopy images\index{Microscopy images}, it is
not enough to just share the pixels\index{Pixels} of the
image. Without contextual
information\index{Contextual~information|textbf} the data is useless.

How do we measure the quality of the contextual information of a
microscope image? To get an answer, we take four steps:
\begin{itemize}
\item Analytical decomposition\index{Analytical~decomposition} of the
  pipeline\index{Pipeline} of data acquisition\index{Data acquisition}
  in microscopy science\index{Microscopy science}.
\item Survey experimental biologists, bioinformaticians, theoretical
  biologists and medical experts.
\item Check existing image repositories\index{Image!Repository} for
  their metadata\index{Metadata}.
\item Check existing metadata\index{Metadata} schemas\index{Schema}
\end{itemize}

\subsubsection{Analytical decomposition}

The Analytical decomposition
step\index{Analytical~decomposition|textbf} is performed by taking a
step backwards and providing an abstract description of the
process. For this process we use an upper
ontology\index{Upper~Ontology|see{Ontology,~Upper}}\index{Ontology!Upper}
to help guide the abstract description. Guided by an overview of upper
ontologies\cite{mascardi2007comparison}, we decided to use the
BFO\cite{grenon2004biodynamic} as our upper ontology, because of its
small size and the research focused design.
\begin{displaymath}
  \xymatrix{ & \text{Biologist} \ar[dl] \ar[d] \ar[dr] \ar@{<->}[rr] &
    & \text{Analyst} \ar[dl] \ar[d] \ar[dr] & \\ \text{Sample} \ar[d]
    \ar[r] & \text{Microscope} \ar[d] \ar[r] & \text{Image} \ar[r] &
    \text{Computer} \ar[d] \ar[r] & \text{Data} \\ \text{Biological
      process} & \text{Interaction} & & \text{Computation} & }
\end{displaymath}
A sample\index{Sample} could be considered both an
object\index{Object} and an
object\_aggregate\index{Object~aggregate}. Because of this dual nature
we will model it as a material\_entity\index{Material~entity}, which
is the superclass of both. A microscope\index{Microscope} and a
computer\index{Computer} are subclasses of
object\index{Object}. Image\index{Image} and data\index{Data} are
subclass of
generically\_dependent\_continuant\index{Generically~dependent~continuant}.

Biological process, (physical) interaction and computation can all be
considered to be subclasses of process.

\subsubsection{Survey experts}

Survey\index{Survey|textbf} experimentalists, analysts and modelers
what the important knowledge is in their domain.

\subsubsection{Existing image repositories}

Examine the existing image repositories\index{Image!Repository} and
their choices in the metadata\index{Metadata} they include.

\subsubsection{Existing metadata schemas}

Check existing metadata schemas\index{Schema|textbf}. This includes
image formats\index{Image!Format} and ontologies\index{Ontology}.

\subsection{Assumed important elements}

I assume the following elements need to be described:
\begin{itemize}
\item Biological origin:
  Species\index{Species}/Cell culture\index{Cell culture} (including
  age), Anatomy structures\index{Anatomy structures}
\item Chemical and physical conditions\index{Chemical
  conditions}\index{Physical conditions}
\item Imaging techniques\index{Imaging techniques}
\end{itemize}

%------------------------------------------------------------------------------
%	Semantics and Reasoning
%------------------------------------------------------------------------------

\section{Semantics and Reasoning}

This section describes the main technology that is used to arrive at
new insights. What do I mean when I talk about
semantics\index{Semantics|textbf} and
reasoning\index{Reasoning|textbf}? This describes the larger field,
with a subsection delving into the semantic web\index{Semantic web}.

%-----------------------------------
%	Semantic Web
%-----------------------------------

\subsection{Semantic Web}

This section describes the Semantic Web\index{Semantic web|textbf}
idea and its main technologies. What are the restrictions, what can we
expect?

The Semantic Web is a concept that revolves around the usage of
different technologies. One of the most important languages used is
that of OWL (Web Ontology Language)\index{OWL}. In this thesis we will
be using the latest version of OWL at the time of writing, which is
OWL 2. we will use Manchester Syntax\index{Manchester Syntax} to
describe the formal structure of our
ontologies\index{Ontology}. Because we want full flexibility in our
modeling\index{Modeling}, we will use OWL 2 Full by using RDF-based
semantics\index{RDF}\index{Semantics}.



%------------------------------------------------------------------------------
%	Annotation and Retrieval
%------------------------------------------------------------------------------

\section{Annotation and Retrieval}

This section will provide information about
annotation\index{Annotation|textbf} of other resources, images in
particular. The classical use for annotation is
retrieval\index{Retrieval|textbf}. Annotation may also be used for
reasoning\index{Reasoning} and pattern discovery\index{Pattern
  discovery}.

%------------------------------------------------------------------------------
%	Life Science Images
%------------------------------------------------------------------------------

\section{Life Science Images}

Life Science Images\index{Life science images|textbf} are images that
are taken in research from the Life Sciences. These range from
brightfield microscopy\index{Brightfield~microscope} to atomic force
microscopy\index{Atomic force microscope}.

%-----------------------------------
%	Developmental biology and model organisms
%-----------------------------------

\subsection{Developmental biology and model organisms}

Developmental biology\index{Developmental biology|textbf} is the study
of developing organisms. From embryo\index{Embryo|see{Stage,
    Embryo}}\index{Stage!Embryo} to
adolescent\index{Adolescent|see{Stage,
    Adolescent}}\index{Stage!Adolescent}, the body goes through a lot
of changes. To study growth patterns and the way growth can be
disrupted by various factors, model organisms\index{Model
  organisms|textbf} are used. Examples of model organisms are: fruit
flies, zebrafish, mice and monkeys.

%-----------------------------------
%	Image Acquisition in the Life Sciences
%-----------------------------------

\subsection{Image Acquisition in the Life Sciences}

Image Acquisition\index{Image!Acquisition|textbf} in the Life Sciences
is done using specialized microscopes. From the regular brightfield
microscope\index{Brightfield~microscope} to FRAP (Fluorescence
recovery after photobleaching)\index{FRAP}. To get an overview of the
field we are going to split it up using some broad categories:
optical, electron and force microscopy.

\subsubsection{Optical microscopy}

Optical microscopy is the domain of microscopy that uses photons as
the main interaction particle with the sample. The simplest technique
that could conceivably be called optical microscopy is observation of
a sample by scanning it in with a flat-back scanner. The magnification
in this case is not really realized with a lens system, but because
the scanning device has more optical resolution than the human eye, we
can still call this technique optical microscopy.

The simplest system that has some form of optical magnification is a
loupe. It also feels a bit like a stretch to call this a
microscope. There seems to be a boundary on the magnification when
something can be called microscopy or not, although this boundary is
not well defined.

The simplest device that is actually called a microscope in use today
is a bright field microscope. In optical microscopy a lot of different
microscopy techniques are named after the way the sample is
illuminated. Bright field microscopy illuminates the sample from below
and detects the light that transmits through the sample. The
background, unhindered by any absorbing material is therefore bright
in the image, hence the name bright field. [TODO: insert image]

Dark field microscopy uses two apertures to filter out the direct
light source, so only light diffracted by the sample appears in the
image. With no diffractive medium, the image will be dark, hence the
name dark field. [TODO: insert image]

The difference between the two microscopes described above (the bright
field and dark field microscopy) can also be described by a difference
of the interaction being detected. In the case of bright field
microscopy, the absorbance of the sample is visualized, while in the
case of dark field microscopy, we observe the diffraction.

Light waves can have a particular polarization. This phase information
can be used in techniques such as phase contrast microscopy and
differential interference contrast microscopy. [TODO: insert image]

The matter can also interact energetically with the light. In this
case the sample absorbs the photons of certain wavelengths and the
molecules are excited. The excited molecules can then fall back to
ground state, emitting photons of a higher wavelength. This process is
called fluorescence. Direct measurement of fluorescence is called
fluorescence microscopy, but there are other possibilities. Different
dynamics of this excitation process lead to other observables used in
FRET, PALM, STORM and TIRF [TODO: Check these techniques].

Confocal microscopy uses double apertures to image a particular
z-slice of a sample. [TODO: insert image]

A common technique to use after optical microscopy is
deconvolution. Optical signals are convoluted by optical distortions,
which can be characterized mathematically and reversed. The process is
computationally intensive, but leads to sharper images. This process
requires a point spread function (PSF), which can be obtained by
measurement or theoretical approximation. For the latter, information
about the lenses is vital.

\subsubsection{Electron microscopy}

Electron microscopy uses electrons to probe the sample. These
microscopes are often more expensive and harder to use than optical
microscopes. The samples need special preparation with metal atoms and
need to be in a vacuum chamber when imaging. These requirements rule
out imaging of living samples.

There are two main types of Electron microscopy: Scanning Electron
microscopy and Transmission Electron microscopy. Scanning Electron
microscopy images the reflection of electrons, while transmission
electron microscopy images the transmission. [TODO: why use one or the other] [TODO: insert images]

Electron microscopy is more precise than optical microscopy, because
the DeBroglie wavelength of electrons is smaller than that of visible
photons. At this resolution different problems might arise, like
thermal noise. That is why some Electron microscopy experiments are
done at extremely low temperatures to reduce this noise. This
technique is usually called Cryo-EM.

\subsubsection{Force microscopy}

Another interaction that can be measured and imaged is atomic force
microscopy. A sample is scanned by going over its surface with a
pin. This pin can rise and fall as a spring pulls it downwards and the
surface of the sample pushes it upwards. The motion of the pin is
measured and in this way a height map of the surface can be made. This
technique can achieve atomic resolution.

\subsubsection{Other}

[TODO: describe PET scanner]
[TODO: describe X-Ray crystallography]

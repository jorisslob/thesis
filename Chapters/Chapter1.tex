% Chapter Template

\chapter{Introduction} % Main chapter title

\label{Chapter1}

\lhead{Chapter 1. \emph{Introduction}} 

%------------------------------------------------------------------------------
%	Overview
%------------------------------------------------------------------------------

\section{General Overview}

The scientific method \index{Scientific method} relies on the
generation of measurements\index{Measurements}. For reproducibility
\index{Reproducibility} measurements should be reported in such a way
that other scientist can replicate the results.

Within computational science the ability to replicate results is
called Reproducible Research\index{Reproducible
  research}\cite{FomelReproducibleResearch}. For the Life
Sciences\index{Life science} this approach is more problematic,
because of the enormous amount of parameters in any given
experiment. A simple workflow\index{Workflow} might look like this:

\begin{itemize}
\item Preparation of sample\index{Sample preparation}
\item Microscopy on sample\index{Microscopy}
\item Image analysis\index{Image!Analysis}
\item Statistical analysis\index{Statistical analysis}
\end{itemize}

Each of these steps deserve a methods page on their own, yet papers
need to communicate their findings succinctly. When there is no room
for in the papers for the provenance\index{Provenance} information,
this information has to stored next to or inside the data itself.

In this thesis I will explore the questions how we can capture, store
and leverage this metadata\index{Metadata}, especially in the area of
microscopy images\index{Microscopy images}.


\subsection{Contextual information}

When talking about microscopy images\index{Microscopy images}, it is
not enough to just share the pixels\index{Pixels} of the
image. Without contextual
information\index{Contextual~information|textbf} the data is useless.

How do we measure the quality of the contextual information of a
microscope image? To get an answer, we take four steps:
\begin{itemize}
\item Analytical decomposition\index{Analytical~decomposition} of the
  pipeline\index{Pipeline} of data acquisition\index{Data acquisition}
  in microscopy science\index{Microscopy science}.
\item Survey experimental biologists, bioinformaticians, theoretical
  biologists and medical experts.
\item Check existing image repositories\index{Image!Repository} for
  their metadata\index{Metadata}.
\item Check existing metadata\index{Metadata} schemas\index{Schema}
\end{itemize}

\subsubsection{Analytical decomposition}

The Analytical decomposition
step\index{Analytical~decomposition|textbf} is performed by taking a
step backwards and providing an abstract description of the
process. For this process we use an upper
ontology\index{Upper~Ontology|see{Ontology,~Upper}}\index{Ontology!Upper}
to help guide the abstract description. Guided by an overview of upper
ontologies\cite{mascardi2007comparison}, we decided to use the
BFO\cite{grenon2004biodynamic} as our upper ontology, because of its
small size and the research focused design.
\begin{displaymath}
  \xymatrix{ & \text{Biologist} \ar[dl] \ar[d] \ar[dr] \ar@{<->}[rr] &
    & \text{Bioinformatician} \ar[dl] \ar[d] \ar[dr] &
    \\ \text{Sample} \ar[d] \ar[r] & \text{Microscope} \ar[d] \ar[r] &
    \text{Image} \ar[r] & \text{Computer} \ar[d] \ar[r] & \text{Data}
    \\ \text{Biological process} & \text{Physical interaction} & &
    \text{Computation} & }
\end{displaymath}
A sample\index{Sample} could be considered both an
object\index{Object} and an
object\_aggregate\index{Object~aggregate}. Because of this dual nature
we will model it as a material\_entity\index{Material~entity}, which
is the superclass of both. A microscope\index{Microscope} and a
computer\index{Computer} are subclasses of
object\index{Object}. Image\index{Image} and data\index{Data} are
subclass of
generically\_dependent\_continuant\index{Generically~dependent~continuant}.

Biological process, physical interaction and computation can all be
considered to be subclasses of process.

\subsubsection{Survey experts}

Survey\index{Survey|textbf} experimentalists, analysts and modelers
what the important knowledge is in their domain.

\subsubsection{Existing image repositories}

Examine the existing image repositories\index{Image!Repository} and
their choices in the metadata\index{Metadata} they include.

\subsubsection{Existing metadata schemas}

Check existing metadata schemas\index{Schema|textbf}. This includes
image formats\index{Image!Format} and ontologies\index{Ontology}.

\subsection{Assumed important elements}

I assume the following elements need to be described:
\begin{itemize}
\item Biological origin:
  Species\index{Species}/Cell culture\index{Cell culture} (including
  age), Anatomy structures\index{Anatomy structures}
\item Chemical and physical conditions\index{Chemical
  conditions}\index{Physical conditions}
\item Imaging techniques\index{Imaging techniques}
\end{itemize}

%------------------------------------------------------------------------------
%	Semantics and Reasoning
%------------------------------------------------------------------------------

\section{Semantics and Reasoning}

This section describes the main technology that is used to arrive at
new insights. What do I mean when I talk about
semantics\index{Semantics|textbf} and
reasoning\index{Reasoning|textbf}? This describes the larger field,
with a subsection delving into the semantic web\index{Semantic web}.

%-----------------------------------
%	Semantic Web
%-----------------------------------

\subsection{Semantic Web}

This section describes the Semantic Web\index{Semantic web|textbf}
idea and its main technologies. What are the restrictions, what can we
expect?

The Semantic Web is a concept that revolves around the usage of
different technologies. One of the most important languages used is
that of OWL (Web Ontology Language)\index{OWL}. In this thesis we will
be using the latest version of OWL at the time of writing, which is
OWL 2. we will use Manchester Syntax\index{Manchester Syntax} to
describe the formal structure of our
ontologies\index{Ontology}. Because we want full flexibility in our
modeling\index{Modeling}, we will use OWL 2 Full by using RDF-based
semantics\index{RDF}\index{Semantics}.



%------------------------------------------------------------------------------
%	Annotation and Retrieval
%------------------------------------------------------------------------------

\section{Annotation and Retrieval}

This section will provide information about
annotation\index{Annotation|textbf} of other resources, images in
particular. The classical use for annotation is
retrieval\index{Retrieval|textbf}. Annotation may also be used for
reasoning\index{Reasoning} and pattern discovery\index{Pattern
  discovery}.

%------------------------------------------------------------------------------
%	Life Science Images
%------------------------------------------------------------------------------

\section{Life Science Images}

Life Science Images\index{Life science images|textbf} are images that
are taken in research from the Life Sciences. These range from
brightfield microscopy\index{Brightfield~microscope} to atomic force
microscopy\index{Atomic force microscope}.

%-----------------------------------
%	Developmental biology and model organisms
%-----------------------------------

\subsection{Developmental biology and model organisms}

Developmental biology\index{Developmental biology|textbf} is the study
of developing organisms. From embryo\index{Embryo|see{Stage,
    Embryo}}\index{Stage!Embryo} to
adolescent\index{Adolescent|see{Stage,
    Adolescent}}\index{Stage!Adolescent}, the body goes through a lot
of changes. To study growth patterns and the way growth can be
disrupted by various factors, model organisms\index{Model
  organisms|textbf} are used. Examples of model organisms are: fruit
flies, zebrafish, mice and monkeys.

%-----------------------------------
%	Image Acquisition in the Life Sciences
%-----------------------------------

\subsection{Image Acquisition in the Life Sciences}

Image Acquisition\index{Image!Acquisition|textbf} in the Life Sciences
is done using specialized microscopes. From the regular brightfield
microscope\index{Brightfield~microscope} to FRAP (Fluorescence
recovery after photobleaching)\index{FRAP}.

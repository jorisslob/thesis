% Chapter Template

\chapter{Introduction} % Main chapter title

\label{Chapter1}

\lhead{Chapter 1. \emph{Introduction}} 

%------------------------------------------------------------------------------
%	Overview
%------------------------------------------------------------------------------

\section{General Overview}

The scientific method \index{Scientific method} relies on the
generation of measurements\index{Measurements}. For reproducability
\index{Reproducability} measurements should be reported in such a way
that other scientist can replicate the results.

Within computational science the ability to replicate results is
called Reproducable Research\index{Reproducable
  research}\cite{FomelReproducibleResearch}. For the Life
Sciences\index{Life science} this approach is more problematic,
because of the enormous amount of parameters in any given
experiment. A simple workflow\index{Workflow} might look like this:

\begin{itemize}
\item Preperation of sample\index{Sample preperation}
\item Microscopy on sample\index{Microscopy}
\item Image analysis\index{Image analysis}
\item Statistical analysis\index{Statistical analysis}
\end{itemize}

Each of these steps deserve a methods page on their own, yet papers
need to communicate their findings succinctly. When there is no room
for in the papers for the provenance\index{Provenance} information,
this information has to stored next to or inside the data itself.

In this thesis I will explore the questions how we can capture, store
and leverage this metadata\index{Metadata}, especially in the area of
microscopy images\index{Microscopy images}.


\subsection{Contextual information}

When talking about microscopy images\index{Microscopy images}, it is
not enough to just share the pixels\index{Pixels} of the
image. Without contextual information\index{Contextual information}
the data is useless.

How do we measure the quality of the contextual information of a
microscope image? To get an answer, we take three steps:
\begin{itemize}
\item Analytical decomposition\index{Analytical decomposition} of the
  pipeline\index{pipeline} of data acquisition\index{Data acquisition}
  in microscopy science\index{Microscopy science}.
\item Survey experimental biologists\index{Experimental biologist},
  bio-informaticians\index{Bio-informatician}, theoretical
  biologists\index{Theoretical biologist} and medical
  experts\index{Medical expert}.
\item Check existing image repositories\index{Image repository} for
  their metadata\index{Metadata}.
\end{itemize}

I assume the following elements need to be described:
\begin{itemize}
\item Biological origin\index{Biological origin}:
  Species\index{Species}/Cell culture\index{Cell culture} (including
  age), Anatomy structures\index{Anatomy structures}
\item Chemical and physical conditions\index{Chemical
  conditions}\index{Physical conditions}
\item Imaging techniques\index{Imaging techniques}
\end{itemize}

%------------------------------------------------------------------------------
%	Semantics and Reasoning
%------------------------------------------------------------------------------

\section{Semantics and Reasoning}

This section describes the main technology that is used to arive at
new insights. What do I mean when I talk about semantics and
reasoning? This describes the larger field, with a subsection delving
into the semantic web.

%-----------------------------------
%	Semantic Web
%-----------------------------------

\subsection{Semantic Web}

This section describes the Semantic Web\index{Semantic web} idea and
its main technologies. What are the restrictions, what can we expect?

The Semantic Web is a concept that revolves around the usage of
different technologies. One of the most important languages used is
that of OWL (Web Ontology Language)\index{OWL}. In this thesis we will
be using the latest version of OWL at the time of writing, which is
OWL 2. we will use Manchester Syntax\index{Manchester Syntax} to
describe the formal structure of our
ontologies\index{Ontology}. Because we want full flexibility in our
modeling\index{Modeling}, we will use OWL 2 Full by using RDF-based
semantics\index{RDF}\index{semantics}.



%------------------------------------------------------------------------------
%	Annotation and Retrieval
%------------------------------------------------------------------------------

\section{Annotation and Retrieval}

This section will provide information about
annotation\index{Annotation} of other resources, images in
particular. The classical use for annotation is
retrieval\index{Retrieval}. Annotation may also be used for
reasoning\index{Reasoning} and pattern discovery\index{Pattern
  discovery}.

%------------------------------------------------------------------------------
%	Life Science Images
%------------------------------------------------------------------------------

\section{Life Science Images}


%-----------------------------------
%	Developmental biology and model organisms
%-----------------------------------

\subsection{Developmental biology and model organisms}


%-----------------------------------
%	Image Acquisition in the Life Sciences
%-----------------------------------

\subsection{Image Acquisition in the Life Sciences}



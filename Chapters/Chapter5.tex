% Chapter Template

\chapter{Use Case: Cytomics Database} % Main chapter title

\label{Chapter5} % Change X to a consecutive number; for referencing this chapter elsewhere, use \ref{ChapterX}

\lhead{Chapter 5. \emph{Cytomics Database}} % Change X to a consecutive number; this is for the header on each page - perhaps a shortened title

%------------------------------------------------------------------------------
%	SECTION 1
%------------------------------------------------------------------------------

\section{Cytomics overview}

The Cytomics database\index{Cytomics database|textbf} is a platform
that stores high-throughput\index{High-throughput} experiments.

%------------------------------------------------------------------------------
%	Annotation for Microscopes
%------------------------------------------------------------------------------

\section{Annotation for Microscopes}

Because of every lab only has a limited amount of
microscopes\index{Microscope}, every microscope can be well
annotated\index{Annotation} and the metadata\index{Metadata} can be
attached to the images it generates.

%-----------------------------------
%	Annotating old Data
%-----------------------------------

\subsection{Annotating old Data}

Before the Imaging Ontology\index{Imaging ontology} was made, the
Cytomics database\index{Cytomics database} already accepted entries. We therefore have to annotate old data\index{Annotating!old data}.

%-----------------------------------
%	Annotating new Data
%-----------------------------------

\subsection{Annotating new Data}

New data that enters the database can be automatically enhanced with
appropriate metadata\index{Metadata}. The
annotation\index{Annotating!new data} of new data is done using
predefined imaging equipment and image header\index{Image!Header}
parsing\index{Parsing}.
